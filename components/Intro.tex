
\section{The Standard Model}
The Standard Model (SM)\cite{SMref1}\cite{SMref2}\cite{SMref3}, forming up since 1970s, is now the fundamental of elementary particle physics. After half a century's development, nowadays the Standard Model is a framework that is confirmed by the observatory of numerous experiments and can be used to explain most of the particle behaviors and interactions in this universe. In the Standard Model there are generally 2 categories of particles: fermions and bosons. Fermions, which always have half-integer spins, make up all the matter in the universe. On the other hand, bosons with integer spins mediate the fundamental interactions among the fermions, and the interactions here includes electro-magnetic interaction, weak interaction and strong interaction. Figure~\ref{fig:smpfamily} shows all the particles that have been discovered and included in the Standard Model.
\begin{figure}[htbp]
\begin{center}
\includegraphics[width=0.72\linewidth]{figures/smpfamily.pdf}
\caption{The Standard Model contains 3 generations' leptons and quarks, 4 kinds of vector bosons and the Higgs boson.}
\label{fig:smpfamily}
\end{center}
\end{figure}

\subsection{Fermions}
Fermions are particles that follow Fermi–Dirac statistics and obey the Pauli exclusion principle. Every fermion has its anti particle, which is a fermion with opposite charge but same mass and spin. The elementary fermions in the Standard Model includes leptons and quarks, both of them consist of 3 generations and each generation consists of 2 flavors of fermions. Every elementary fermion in the Standard Model has half spin.
\subsubsection{Lepton}
The leptons are the fermions that never participate in strong interaction. There are 3 generations of leptons, and each generation consists of two flavors of leptons: one charged lepton such as electrons, and one neutral lepton also known as "neutrino". The charged letpon always carries 1 unit of elementary electric charge, negative for a lepton and possitive for its anti-lepton. The charged leptons have their masses and the masses have been precisely measured as shown in Figure~\ref{fig:smpfamily}. In SM, neutrinos are regarded to be massless.

\vspace{0.3cm}
The first generation of leptons includes electron($e^{-}$) and electron-neutrino($\nu _{e}$). Both electron and electron-neutrino have an electron number $L_{e}=1$, while their anti leptons have $L_{e}=-1$. The second generation leptons, including muon($\mu$) and muon-neutrino($\nu_{\mu}$), have a muon number $L_{\mu}=1$, while $L_{\mu }=-1$ for anti-muon($\bar{\mu}$) and anti-muon-neutrino($\bar{\nu} _{\mu }$). Similarly, the third generation consists of tau($\tau$) and tau-neutrino($\nu _{\tau }$), with a tau number $L_{\tau}$ accordingly.

\vspace{0.3cm}
In SM, under the assumption that neutrinos are massless, the lepton numbers are strictly conserved in any kind of interactions. However, recent experiments\cite{neutrinoOscillation1}\cite{neutrinoOscillation2} indicate that the neutrinos have small masses, which implies the lepton numbers can be mixed among different generations,  with small cross-sections though.
\subsubsection{Quark}
A quark can participate in any of the 3 interactions in Standard Model. Like leptons, quarks fall into 3 generations, and each generation contains 2 flavors of quarks which each have 2/3 and -1/3 elementary electric charge correspondingly. Besides electric charge, quarks also have another intrinsic property known as color charge. The color charge a quark carries can be red, blue or green, while the anti quark carries a corresponding anti-color. Quarks are never observed directly in isolated state due to the phenomenon of color confinement, which confines quarks to only exists in composite colorless particles known as hadrons.

\vspace{0.3cm}
The first generation includes up and down quarks, the second generation includes charm and strange quarks, and the third generation consists of top and bottom quarks. Every quark has its anti quark, with opposite charge of itself. The mass for each quark is shown in Figure~\ref{fig:smpfamily}. It is possible for heavier quarks to decay into lighter quarks through weak interaction, especially for quarks within the same generation.
\subsection{Bosons and the interactions}
In contrast to fermions, bosons are particles that follow Bose–Einstein statistics and alows multi particles in the same state. In SM there are 4 kinds of vector bosons each with spin 1 and one spin-0 scalar boson which is the lately discovered Higgs boson. The vector bosons work as force carriers of the 3 interactions, and the Higgs boson deliever masses to elementary particles by interacting with them.
\subsubsection{Vector Bosons}
In SM, the vector bosons work as mediations of the 3 fundamental interactions among fermions, including gluon, photon, Z boson and W boson. Each of the vector bosons has spin 1.
\begin{itemize}
\item \textbf{gluon} is the force mediation of strong interaction, described in Quantum Chromodynamics (QCD) theory, a gauge theory based on SU(3). Gluons are massless and has no electric charge.
\item \textbf{photon} is the force mediation of electromagnetic interaction, described in Quantum Electrodynamics (QED) theory, a $U(1)_{EM}$ theory. Photons are massless and has no electric charge.
\item \textbf{Z boson} is the mediation of weak interaction with no electric charge flow. The Z boson has no charge, while it has mass of 91.2 GeV, which makes it possible to decay into a pair of fermion–antifermion.
\item \textbf{W boson} is the mediation of weak interaction with electric charge flow. The W carries either 1 or -1 elementary electric charge, noted as $W^{+}$ and $W^{1}$, both having mmasses of 80.4 GeV. Like the Z boson, W bosons can decay into fermion–antifermion pairs.
\end{itemize}
In SM, the unification of electromagnetic interaction and weak interaction is realized by an SU(2) $\times$ U(1) gauge group. Photon, Z boson and W boson are the productions of the SU(2) $\times$ U(1) group due to spontaneous symmetry breaking caused by the Higgs mechanism. 
\subsubsection{The Higgs}
In the 1960s, the Higgs boson and Higgs mechanism is proposed in gesture to explain the source of gauge bosons' masses\cite{higgstheory1}\cite{higgstheory2}\cite{higgstheory3}, as the gauge bosons are believed to be massless according to the gauge theory, which conflicts with the experiment facts. It indicates that the massless gauge bosons interact with a scalar field which leads to a spontaneous symmetry breaking and finally gives mass to the gauge bosons. The scalar field here is the Higgs field. 

\vspace{0.3cm}
On July 4 of 2012, the god particle Higgs boson was announced to be discovered by both CMS and ATLAS experiments at LHC, CERN\cite{higgsdiscover1}\cite{higgsdiscover2}. Hence the Higgs boson officially became one member of the standard model elementary particle family. The Higgs boson has no spin or electric charge, and its mass is measured to be 125GeV. It is very unstable and mainly decays into b$\bar{b}$ quark pair, $\tau\bar{\tau}$ pairs or gauge boson pairs.

\section{The Limitation of Standard Model}
Though the SM has been tested and demonstrated as a great success among numerous particle physics experiments and provided reliable physics predictions for most of the sceneries, it is not yet believed to be a complete theory. The SM does not provide theoretical support for either dark energy or dark matter particles which are believed to exist according to cosmological observations. Moreover, within the SM particles, neutrino oscillation proved by several experiments conflicts with the SM's assumption of massless neutrinos. Above all, as mentioned above, the SM incorporates only 3 of the 4 fundamental interactions, leaving gravitation completely unexplained in the scope of particle physics.
\subsubsection{The Hierarchy problem}
The Hierarchy problem in particle physics refers to the huge discrepancy between the electroweak scale and the gravitional scale, as the weak force is $10^{24}$ times stronger than gravity. The Fermi's constant denoting the weak interaction is expected to be larger and closer to the Newton's constant for gravity, based on the calculation of SM. In other words, we expect the large quantum contributions to the square of the Higgs boson mass would make the Higgs boson much heavier than the measured 125GeV. It could be that the measured Higgs boson mass is the result of incredibly fine-tuned constants within the SM. And alternatively, some new theoretical mechanics is expected, among which the Bulk RS Graviton Model is one of the possible solution.

\section{Bulk RS Graviton Model}
%\subsection{Introduction of Extra Dimension Models}

%\subsection{Randall–Sundrum Model}
%\subsection{The Bulk RS Graviton Model}
%\subsection{Phenominology of the Bulk Graviton Model}

\section{Status of Searches for the Bulk Graviton}
As described above, the Bulk Graviton can be produced by gluon-gluon fusion as well as quark-quark fusion, which means that the proton-proton collisions produced in the Large Hadron Collider can be a good source of the Bulk Graviton, if existing. 
\subsection{Previous Searches}
Attempts of searching~\cite{Aad:2012nev,Aad:2013wxa,Aad:2014xka,Chatrchyan:2012baa,Khachatryan:2014gha,Aaboud:2016okv} for the Bulk Graviton have been made based on data collected from both ATLAS and CMS detectors with the proton-proton center of mass energy at 7, 8 and 13 GeV. The limits on the cross section for the production of Bulk Graviton have been set as a function of $m_{G}$. The existence of the Bulk Graviton was exluded with mass below 610 GeV~\cite{Chatrchyan:2012baa} for $\tilde{k}=0.5$ and mass below 1100 GeV~\cite{Aaboud:2016okv} for $\tilde{k}=1.0$,at the confidence level of 95\%. As one can see from these searches, semi-leptonic final state is the most popular channel. And there is no previous search exisiting looking into the channel of ZZ to $2\ell 2\nu$.
\subsection{$2\ell 2\nu$ channel and  Search Strategy}
In this analysis we present a search for the Bulk Graviton resonance decaying into a pair of Z bosons, in which one of the Z boson decays into a pair of charged leptons, either electron pair or muon pair(denoted by "$\ell$"), while the other into two neutrinos(denoted by "$\nu$"). Figure~\ref{fig:intro_llnndiagram} shows the Feynman diagram of this process.
\begin{figure}[htbp]
\begin{center}
\includegraphics[width=0.72\linewidth]{figures/intro_llnndiagram.pdf}
\caption{Leading order Feynman diagram for the production of resonance X via gluon–gluon fusion decaying to the ZZ $2\ell 2\nu$ final state}
\label{fig:intro_llnndiagram}
\end{center}
\end{figure}
The $2\ell 2\nu$ channel has its signal signiatures of a pair of adjacent leptons from a boosted Z boson decaying and high ${p_{T}}^{missing}$ from the other Z boson which decays to neutrinos.

\vspace{0.3cm}
Comparing to the semi-leptonic channel, the $2\ell 2\nu$ channel has its advantage in that the background is less and easier to model. Though Z+jets is a main source of background for both of these two channels, as for semi-leptonic one, it is almost impossible to distinguish the ZZ to $2\ell 2q$ process from Z+jets background, as they have very similar kinematics in physics -- a Z boson and corresponding hadronic recoil. But unlike the Z+jets backgrounds, events from the ZZ to $2\ell 2\nu$ process always come with large ${p_{T}}^{missing}$, which makes it possible to strongly suppress the Z+jets background. Although the $2\ell 2q$ channel has larger branching fraction, the branching fraction of the $2\ell 2\nu$ is still considerable, about 1/3 of that of the $2\ell 2q$ channel and 6 times as large as the fraction of the four charged-lepton final state. This fact might put the $2\ell 2\nu$ channel into disadvantaged position in LHC Run I when we have lower energy and more limited statistics, which explained that no previous search in $2\ell 2\nu$ was carried out. But now in RunII with larger cross section benefited from higher energy and the larger and growing statistics, advantage would be expected from the $2\ell 2\nu$ channel.

\vspace{0.3cm}
Because of the invisible neutrinos from the Z boson decay, we will not have full momentum information of that Z boson, instead, only ${p_{T}}^{missing}$ is accessible, which can be seen as the projection of the invisible Z boson's momentum on the x-y detector plane. Therefore it is not possible to reconstruct the invarient mass of the $2\ell 2\nu$ system. In this analysis the transverse mass ($m_{T}$) is designed and calcuated as the discriminating variable to separate signal over background. The transverse mass variable is defined as Equation~\ref{eqn:MT}:
\begin{equation}
m_{T}^2 = \left[ \sqrt{({p_{T}}^{\ell\ell})^2 + m^2_{\ell\ell}}
      + \sqrt{({p_{T}}^{missing})^2+m^2_{\ell\ell}}\right]^2
      - \left[\vec{p}_{T}^{\ell\ell}+\vec{p_{T}}^{missing}\right]^2,
\label{eqn:MT}
\end{equation}

Here ${p_{T}}^{\ell\ell}$ and $m_{\ell\ell}$ each represent the $p_{T}$ and mass of the Z boson constructed from the charged lepton pair system. the $m_{\ell\ell}$ in the middle term provides an estimator of the mass of the invisibly decaying Z boson. This choice has negligible impact on the expected signal at large $m_{T}$, but is found to preferentially suppress backgrounds from tt and WW decays.

\vspace{0.3cm}
Though the $m_{T}$ variable is designed to work as the invarient mass projected on the 2-dimension plane, a kinematic edge is still expected from the putative heavy resonance, and the edge position strongly depends on the mass of the resonance. Figure~\ref{fig:intro_mt} gives an example of what the transverse mass spectrum looks like from the decay of resonances with invariant mass of 1000 GeV.

\begin{figure}[htbp]
\begin{center}
\includegraphics[width=0.72\linewidth]{figures/intro_example_mt.png}
\caption{The transverse mass($m_{T}$) spectrum of resonances with mass at 1000 GeV.}
\label{fig:intro_mt}
\end{center}
\end{figure}


\section{CMS and Me} %what did i do in this analysis and others
In August 2014, I joined the UVA CMS group and worked on the Shashlik detector muon calibration as my first project here. Aferwards, in the middle of 2015, I left for CERN, Switzerland and stayed there for 2 years. During my stay at CERN I was involved in several projects and activities both on the detector and physics analysis. 

\vspace{0.3cm}
My most detector-related work are around the Hadron Calorimeter(HCAL). Starting with the Hadron Forward(HF) detector frontend electronics Phase I upgrading in 2015, I joined the HCAL upgrading team, helping designing and carring out a series of electronics tests. In 2016 I helped the HCAL Endcap(HE) upgrading with the upgrading system monitoring, and HCAL DQM group with their online DQM system for HF/HE upgrading. Besides, I was taking the HCAL detector on call expert shifts continually for 1 year and a half since 2016.

\vspace{0.3cm}
In terms of physics analysis, I started working on this di-boson analysis in the December of 2015, helping design and implement the analysis framework from scratch, studying the pileup reweighting, trigger efficiency, lepton ID/Iso choices and their efficiencies as well as the non-resonant background data-driven modeling. The muon tracker High $p_{T}$ ID efficiency calculated by me has been widely used in related CMS analyses and I am also contributing to the high $p_{T}$ muon paper carried out by the muon Physics Object Group(POG). Apart from this di-boson analysis, I also worked as the generator contact person for the Beyond Two Generations (B2G) Physics Analysis Group (PAG) for the whole year of 2017.
