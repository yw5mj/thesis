\begin{center}
\textbf{\huge Preface}
\end{center}
\begin{flushright}
\Large{--- CMS and me}
\end{flushright}
\vspace{0.5cm}
In August 2014, I joined the UVA CMS group and worked on calibrating a prototype Shashlik calorimeter using the muon data as my first project here. Afterwards, in the middle of 2015, I left for CERN, Switzerland and stayed there for 2 years. During my stay at CERN I was involved in several projects and activities both on the detector and physics analysis. 

\vspace{0.3cm}
My most detector-related work focused on the Hadron Calorimeter (HCAL). Starting with the Hadron Forward (HF) detector frontend electronics Phase I upgrade in 2015, I joined the HCAL upgrade team, helping design and carry out a series of electronics tests. In 2016 I helped the HCAL Endcap (HE) upgrade with system monitoring and the HCAL Data Quality Management (DQM) group with their online DQM system for the HF/HE upgrades. I was also taking HCAL detector on call expert shifts continually for 1 year and a half since 2016.

\vspace{0.3cm}
In terms of physics analysis, I started working on this di-boson analysis~\cite{thispaper} in the December of 2015, helping to design and implement the analysis framework from scratch, studying the pileup reweighting, trigger efficiency, lepton identification algorithms and their efficiencies, as well as determining the non-resonant background using data-driven modeling. The muon tracker High $p_{T}$ efficiency calculated by me has been widely used in related CMS analyses and I am also contributing to the high $p_{T}$ muon paper carried out by the muon Physics Object Group (POG). Apart from this di-boson analysis, I also worked as the Monte Carlo/Generator contact person for physics simulations in the Beyond Two Generations (B2G) Physics Analysis Group (PAG) for the whole year of 2017.
