miThe background modeling are strategies used to model the backgrounds that matches the data distribution in the control region. The main sources of SM background in this analysis can be divided into 3 groups:
\begin{enumerate}
\item \textbf{Zjets events}: Zjets process is the dominating background source for this search, with $\ell$ in the final state from the Z decay. ${p_{T}}^{miss}$ is only instrumental.
\item \textbf{Non-resonant events}: Events that have $\ell$ and ${p_{T}}^{miss}$ in the final state, but $\ell$ pair does not come from the decay of a resonance. $t\bar{t}$ and $WW$ processes are the main sources of the non-resonant background
\item \textbf{Resonant event}: Events that have $\ell$ and ${p_{T}}^{miss}$ in the final state, while $\ell$ pair comes from the decay of a Z boson. $ZZ$ and $WZ$ are the main sources of the resonant background
\end{enumerate}

\vspace{0.3cm}
The background modeling starts from tuning the MC simulation samples. MC simulation is widely used in this analysis. The modeling of the resonant background and the signal completely rely on MC simulation. Though the modeling of the Zjet and non-resonant background are done by data-driven methods, simulation samples are also used in the process. Various weights are asigned to the MC events to compensate the discrepancies between the simulation settings and the actual experiment, in terms of pileup, HLT and lepton ID/Iso. Then the detail of the background strategies for the 3 background groups are discussed.

\section{Pileup re-weighting}
The number of pileup interactions is directly related to the contamination in an event, and can affect the quality of the event reconstruction, such as the value of ${p_{T}}^{miss}$ and lepton isolations. For the MC events, a random number of pileups are added during their production, which does not necessarily agree with the pileup distribution in the data. To align the distribution of pileup numbers between data and simulation, pileup re-weighting is applied to the MC samples. 

\vspace{0.3cm}
Considering the reconstruction efficiencies of the vertices might differ between MC and data, a safe way is to reweight based on the distribution of the number of actual pileup interactions (true pileup), rather than observed interactions. And the value of the weight assigned to the MC events would be obtained from the ratio of true data pileup profile and true MC pileup profil. The true MC pileup profile can be found in the configuration of the RunIISummer16 MC production. The true data pileup profile can be calculated based on the instantaneous luminosity for each bunch crossing noted in the CMS pileup JSON file, and the recommended Mini-Bias cross-section of $69.2 mb \pm 4.6\%$ evaluated by the CMS luminosity POG.

\vspace{0.3cm}
Figure~\ref{fig:bg_pileup} shows the distribution of the true pileup number for both data and MC, as well as their ratio. The pileup reweighting value for each event in the MC samples are obtained from the ratio plot based on the true pileup number in that event.

\begin{figure}[htbp]
\begin{center}
\includegraphics[width=0.45\linewidth]{figures/bg_Npileup.png}
\includegraphics[width=0.45\linewidth]{figures/bg_pileupratio.png}
\caption{Pileup number profiles (left) for RunIISummer16 MC and 2016 data; and the pileup re-weighting function (right).}
\label{fig:bg_pileup}
\end{center}
\end{figure}

\section{Monte Carlo Efficiencies}
Weights are also applied to the MC samples in terms of the discrepancies in HLT, lepton ID/Iso efficiencies between data and MC samples. The chance of a lepton passing HLT, ID and Iso should be:
\begin{align*}
P(HLT\cdot ID\cdot Iso) & = P(HLT|ID\cdot Iso)\times P(ID\cdot Iso) \\
 & = P(HLT|ID\cdot Iso)\times P(Iso|ID)\times P(ID)
\end{align*}

where $P(HLT\cdot ID\cdot Iso)$ denotes the efficiency of a lepton passing HLT, ID and Iso; $P(HLT|ID\cdot Iso)$ is the efficiency of a lepton passing HLT under the condition of it passing ID and ISO, and is also referred to as trigger efficiency; $P(ID\cdot Iso)$ is the ID/Iso efficiency; $P(Iso|ID)$ is the efficiency of a lepton passing Iso under the condition that it passes ID, and it is also referred to as the Iso efficiency; $P(ID)$ is the ID efficiency. The efficiencies are all calculated using a Tag-and-Probe method in this analysis, as discussed below. 

\subsection{Muon ID/Iso Efficiency}
The ID/Iso efficiencies may differ between MC and data due to detector effect. Therefore efficiencies are measured and scale factors (SF) are calculated for the MC samples to counter this effect. The Muon High $p_T$ ID, Tracker High $p_T$ ID and the Tracker Isolation efficiencies are measured separatedly using the tag-and-probe method described as below. Events are selected from the SingleMuon dataset for data and DYJetsToLL\_M-50\_TuneCUETP8M1\_13TeV-madgraphMLM-pythia8 dataset for MC with pileup reweight. 

\vspace{0.3cm}
In the tag-and-probe method, a tag muon and a probe muon are selected from a event. The tag muon is a muon known as well-defined, which helps eliminate bias and calculate efficiency, and the probe muon is the muon candidate from which the efficiency is studied. A event is kept only if two muons are found with one passing the "tag" criteria while the other passing the "probe" criteria. For the muon ID/Iso efficiency measurement, the tag criteria are:
\begin{enumerate}
\item passing the \texttt{tight} Muon ID and the $relIso<0.2$ Iso requirment recommended by CMS Muon POG
\item passing \texttt{HLT\_IsoMu24} and $p_T > 26 GeV$
\end{enumerate}

The ID and Iso requirment ensures the tag muon to be a well defined muon. The \texttt{HLT\_IsoMu24} is the un-prescaled HLT with lowest $p_T$ threshold in the SingleMuon dataset. Requiring the tag muon to pass this HLT assures that the selected event would be kept in the SingleMuon dataset so that the efficiencies to be masured from the probe muon would not be biased due to the dataset for selection. The $p_T > 26 GeV$ requirment is added to be consistent with the 24 GeV $p_T$ threshold in the HLT. A muon candidate would be considered as a probe if it satisfies the condition of the efficiency measurement. 

\vspace{0.3cm}
Due to a known issue with the tracker system existing during Run B to F caused by heavily ionizing particles (HIPs) and fixed later, some tracking inefficiency is observed. Though the effect on the muon reconstruction is minor, the data efficiency for muons are calculated separatedly for Run B to F and Run G to H. And an additional 1\% uncertainty is considered to cover the effect of the tracking inefficiency issue on the muon reconstruction. Considering 19.71 fb$^{-1}$ in run B to F and 16.15 fb$^{-1}$ in Run G to H, a flat random number between [0,1] is generated for each MC event, and if it is larger than $19.71 fb^{-1}/(19.71 fb^{-1}+16.15 fb^{-1})=0.5496$ the scale factors calculated from the muon efficiency in Run B to F would be assigned to the event, otherwise the SFs from Run GH are assigned.

\subsubsection{Muon ID Efficiency}
In the ID efficiency measurement, any muon candidate from the object reconstruction with $p_T > 20GeV$ and $|\eta|<2.4$ (base muon selection in the analysis) could be considered as a probe muon. Spectrum sets of the invariant masss of the tag-probe muon pairs are calculated for various $p_T - \eta$ bins, with each set including a spectrum for those pairs with probe muon passing the ID, while another spectrum for pairs with probe muon failing the ID. The invariant mass spectrum consists of $Z\rightarrow \mu\mu$ process and other backgrounds. The $Z\rightarrow \mu\mu$ events are considered signal and the probe muon in these events are true muons. In this case, the efficiency would be 
\begin{equation}
\epsilon=N_{signal}^{pass}/(N_{signal}^{pass}+N_{signal}^{fail}). 
\end{equation}
The spectrums are fit in a $signal+background$ pattern, and the integral of the signal shape is $N_{signal}^{pass}$ in the passing category and $N_{signal}^{fail}$ in the failing category for each bin. 

\vspace{0.3cm}
The signal function used for the fitting is the sum of 2 Voigtians. The background profile can either be RooCMSShape~\cite{bg_cmsshape} or third order Chebychev polynomial. An example of the fitting plots are shown in Figure~\ref{fig:bg_tnpmuonid}, for the Tracker High $p_T$ muon ID efficiency measurement.
\begin{figure}[htbp]
\begin{center}
\includegraphics[width=0.9\linewidth]{figures/bg_tnpmuonid.png}
\caption{An example of one $p_T - \eta$ bin of the $\mu\mu$ invariant mass spectrum with the $signal+background$ fittings for the efficiency measurement of the Tracker High $p_T$ ID.}
\label{fig:bg_tnpmuonid}
\end{center}
\end{figure}

\vspace{0.3cm}
Figure~\ref{fig:bg_muontkideff} to \ref{fig:bg_muonmcideff} shows the efficiency results calculated from both Tracker High $p_T$ ID and High $p_T$ ID.

\begin{figure}[htbp]
\begin{center}
\includegraphics[width=0.49\linewidth, page=1]{figures/bg_muonidisoeff.pdf}
\includegraphics[width=0.49\linewidth, page=2]{figures/bg_muonidisoeff.pdf}
\caption{High $p_T$ Muon ID efficiency for 2016 ReReco data as a function of muon $p_T$ and $|\eta|$, for 2016 B-F (left) and 2016 GH (right).}
\label{fig:bg_muontkideff}
\end{center}
\end{figure}

\begin{figure}[htbp]
\begin{center}
\includegraphics[width=0.49\linewidth, page=3]{figures/bg_muonidisoeff.pdf}
\includegraphics[width=0.49\linewidth, page=4]{figures/bg_muonidisoeff.pdf}
\caption{Tracker High $p_T$ Muon ID efficiency for 2016 ReReco data as a function of muon $p_T$ and $|\eta|$, for 2016 B-F (left) and 2016 GH (right).}
\label{fig:bg_muonideff}
\end{center}
\end{figure}

\begin{figure}[htbp]
\begin{center}
\includegraphics[width=0.49\linewidth, page=5]{figures/bg_muonidisoeff.pdf}
\includegraphics[width=0.49\linewidth, page=6]{figures/bg_muonidisoeff.pdf}
\caption{Muon ID efficiency for RunIISummer16 MC as a function of muon $p_T$ and $|\eta|$, for High $p_T$ Muon ID (left) and Tracker High $p_T$ Muon ID (right).}
\label{fig:bg_muonmcideff}
\end{center}
\end{figure}

\vspace{0.3cm}
As the Tracker High $p_T$ ID is a loosen version of the High $p_T$ ID. if a muon passes the High $p_T$ ID, it would pass the Tracker High $p_T$ ID too. Then the muon ID scale factor for an event would be calculated as:
\begin{small}
\begin{align*}
SF & =\epsilon^{data}/\epsilon^{MC} \\
 & =\frac{(\epsilon_{HighPt}(\mu_1)\times \epsilon_{trkHighPt}(\mu_2)+\epsilon_{trkHighPt}(\mu_1)\times \epsilon_{HighPt}(\mu_2)-\epsilon_{HighPt}(\mu_1)\times \epsilon_{HighPt}(\mu_2))_{data}}{(\epsilon_{HighPt}(\mu_1)\times \epsilon_{trkHighPt}(\mu_2)+\epsilon_{trkHighPt}(\mu_1)\times \epsilon_{HighPt}(\mu_2)-\epsilon_{HighPt}(\mu_1)\times \epsilon_{HighPt}(\mu_2))_{MC}}
\end{align*}
\end{small}

\subsubsection{Muon Iso Efficiency}
The muon tracker isolation efficiency is also measured using the tag-and-probe method, with an additional requirment on the probe muon to pass the tracker High $p_T$ ID. As the tracker isolation selection applies to both muons, the ratio between the efficiencies of data and MC is calculated as the scale factor of the isolation. The SF values versus $p_T - \eta$ are show in Figure~\ref{fig:bg_muonisosf}
\begin{figure}[htbp]
\begin{center}
\includegraphics[width=0.49\linewidth, page=7]{figures/bg_muonidisoeff.pdf}
\includegraphics[width=0.49\linewidth, page=8]{figures/bg_muonidisoeff.pdf}
\caption{\texttt{tracker ISO} data/MC efficiency scale factors a function of muon $p_T$ and $|\eta|$, for 2016 B-F (left) and 2016 GH (right).}
\label{fig:bg_muonisosf}
\end{center}
\end{figure}

\subsection{Electron ID/Iso Efficiency}
Based on the recommendation from EGamma POG, the \texttt{Loose} cut-based identification (ID) selection is required for all the electron candidates. As a PF isolation is already included in the \texttt{Loose} ID, no additional electron isolation is needed.

\vspace{0.3cm}
The electron \texttt{Loose} ID (including PF ISO) efficiency and scale factors are provided by the EGamma POG, measured by the tag-and-probe method. Figure~\ref{fig:bg_eidsf} shows the electron \texttt{Loose} ID scale factors from EGamma POG used in this analysis. The electron reconstruction is also affected by the tracking inefficiency issue, the reconstruction scale factors are also provided by the EGamma POG to counter this effect. Figure~\ref{fig:bg_gsfsf} shows the electron reconstruction scale factors from EGamma POG used in this analysis.  

\begin{figure}[htbp]
\centering
\includegraphics[width=0.66\linewidth, page=1]{figures/bg_elooseideff.pdf}
\includegraphics[width=0.66\linewidth, page=2]{figures/bg_elooseideff.pdf}
\caption{EGamma POG electron \texttt{Loose} ID (including pf Iso) efficiency scale factors for 2016 dataset analysis.}
\label{fig:bg_eidsf}
\end{figure}

\begin{figure}[htbp]
\centering
\includegraphics[width=0.66\linewidth, page=2]{figures/bg_erecoeff.pdf}
\caption{EGamma POG electron reconstruction scale factors for 2016 dataset analysis.}
\label{fig:bg_gsfsf}
\end{figure}

\subsection{Trigger Efficiency}\label{sec:bkg_trig}
The HLT is designed serving the purpose of fast decision making on data taking, and therefore the reconstruction of objects' properties on the HLT level are not precise enough compared to the offline reconstruction. Trigger efficiency study is important to physics analyses in two aspects: to suppress the trigger efficiency effect on the data by optimizing data selection; and to compensate the discrepancy in terms of trigger efficiency between data and simulation samples by applying trigger efficiency SFs to MC samples.
\subsubsection{SingleMuon HLT Efficiency}
For the muon channel selection, A event are required to pass the single muon HLT of either \texttt{HLT\_Mu50} or \texttt{HLT\_TkMu50}. The combined trigger efficiencies and SFs are centrally derived by the Muon POG, measured with the tag-and-probe method. The summary plots in Figure~\ref{fig:bg_trgeff_mu} shows the trigger efficiencies versus $p_T$ (left) and $eta$ distribution. In the $p_T$ plot, a rising edge of the efficiency can be clearly seen around the $p_T$ threshold of the HLT. To avoid events falling on the rising edge in the trigger efficiency and causing unnecessary challenges in the background modeling, the $p_T$ selection of the leading muon is set to be 60 GeV.

\begin{figure}[htpb]
\begin{center}
\includegraphics[width=0.49\linewidth, page=1]{figures/bg_muontrgeff.pdf}
\includegraphics[width=0.49\linewidth, page=2]{figures/bg_muontrgeff.pdf}
\caption{Muon trigger efficiency for 2016 data and MC, versus $p_T$ (left) and $eta$ (right)}
\label{fig:bg_trgeff_mu}
\end{center}
\end{figure}

\vspace{0.3cm}
The efficiency results are delivered in the form of $p_T - \eta$ 2D histograms. The calculated ratio of effiicencies between data and MC are applied on the leading muons of the pair in the MC events only as the trigger efficiency scale factor, considering very negligible number of events in either data or MC are seen with the subleading muon passing the HLT while leading muon failing.

\subsubsection{SingleElectron HLT Efficiency}
The SingleElectron HLT (\texttt{HLT\_Ele115\_CaloIdVT\_GsfTrkIdT}) is measured for data and MC using the tag-and-probe method. Events are selected from SingleElectron dataset for data and DYJetsToLL\_M-50\_TuneCUETP8M1\_13TeV-amcatnloFXFX-pythia8 dataset for MC with pileup reweight. The "tag" criteria are:
\begin{enumerate}
\item passing the \texttt{tight} cut-based identification (ID) and isolation (Iso) recommended by the EGamma POG %shown in Table~\ref{tab:electron-tightid}
\item passing \texttt{HLT\_Ele27\_WPTight\_Gsf} with $p_T>30$ and $|\eta|<2.1$
\end{enumerate}

%\begin{table}[htb!]
%  \center
%  \caption{The cuts used in the POG \texttt{tight} electron identification.}
%  \label{tab:electron-tightid}
%  \begin{tabular}{r c c c}
%    \hline
%    Variable & Barrel & Endcap \\
%    \hline
%    $|\eta_{\rm SC}|$ acceptance & $(0, 1.479)$ & $(1.479, 2.5)$\\
%    $\sigma_{i\eta,i\eta} <$ & 0.00998  & 0.0292 \\
%    $|\Delta\eta_{in}| <$ & 0.00308  & 0.00605 \\
%    $\Delta\phi_{in} <$ & 0.0816  & 0.0394 \\
%    \texttt{hOverE} $<$ & 0.0414  & 0.0641 \\
%    \texttt{relIsoWithEA} $<$ & 0.0588  & 0.0571 \\
%    $|1/E - 1/p| <$ & 0.0129  & 0.0129 \\
%    expectedMissingInnerHits $\leq$ & 1  & 1 \\
%    conversion veto & yes  & yes \\
%    \hline
%  \end{tabular}
%\end{table}

The criterion of \texttt{tight} ID/Iso ensures the tag electron to be a well defined electron. The \texttt{HLT\_Ele27\_WPTight\_Gsf} is the un-prescaled HLT with lowest $p_T$ threshold in the SingleElectron dataset, and ensures that the trigger efficiency to be measured from the probe electron would not be biased due to the dataset for selection. A electron would be considered as a probe if it satisfies the condition of the trigger efficiency measurement, which is passing the \texttt{loose} ID/Iso in this analysis.

\vspace{0.3cm}
The electron trigger efficiencies are measured as a function electron reconstructed $p_T$ and $|\eta|$ for the 2016 full dataset and RunIISummer16 MC. Figure~\ref{fig:bg_etrgtnp} gives an example of invariant mass spectrum of the electron pair with fitting. As only electrons passing ID/Iso criteria are selected, the background fraction is very small for the trigger efficiency measurement.

\begin{figure}[htpb]
\begin{center}
\includegraphics[width=0.95\linewidth, page=1]{figures/bg_etrgtnp.png}
\caption{An example of one  $p_T - |\eta|$ bin of the electron pair invariant mass spectrum with the $signal+background$ fittings for the efficiency measurement of \texttt{HLT\_Ele27\_WPTight\_Gsf}}
\label{fig:bg_etrgtnp}
\end{center}
\end{figure}

\vspace{0.3cm}
The measured efficiencies are shown in Figure ~\ref{fig:trgeff_el_dt} and Figure~\ref{fig:trgeff_el_mc} for data and MC respectively. The corresponding Data/MC scale factors are shown in Figure~\ref{fig:trgeff_el_sf}. Simular to the muon channel, the SingleElectron HLT efficiency SF is applied on the leading electron in the MC samples as reweighting factors to model the electron trigger efficiency, and $p_T > 120GeV$ selection is added to the leading electron.

\begin{figure}[htpb]
\begin{center}
\includegraphics[width=0.49\linewidth, page=1]{figures/hlt115electron_2016fulleff_absetapt.pdf}
\includegraphics[width=0.49\linewidth, page=2]{figures/hlt115electron_2016fulleff_absetapt.pdf}
\caption{Electron trigger efficiency from 2016 ReReco dataset as a function of reconstructed electron $p_T$ and $|\eta|$. Left for $p_T <150 GeV$, right for $p_T >150GeV$. }
\label{fig:trgeff_el_dt}
\end{center}
\end{figure}

\begin{figure}[htpb]
\begin{center}
\includegraphics[width=0.49\linewidth, page=3]{figures/hlt115electron_2016fulleff_absetapt.pdf}
\includegraphics[width=0.49\linewidth, page=4]{figures/hlt115electron_2016fulleff_absetapt.pdf}
\caption{Electron trigger efficiency from RunIISummer16 MC as a function of reconstructed electron $p_T$ and $|\eta|$. Left for $p_T <150 GeV$, right for $p_T >150GeV$. }
\label{fig:trgeff_el_mc}
\end{center}
\end{figure}

\begin{figure}[htpb]
\begin{center}
\includegraphics[width=0.49\linewidth, page=5]{figures/hlt115electron_2016fulleff_absetapt.pdf}
\includegraphics[width=0.49\linewidth, page=6]{figures/hlt115electron_2016fulleff_absetapt.pdf}
\caption{Electron trigger efficiency Data/MC scale factors as a function of reconstructed electron $p_T$ and $|\eta|$. Left for $p_T <150 GeV$, right for $p_T >150GeV$. }
\label{fig:trgeff_el_sf}
\end{center}
\end{figure}

\clearpage
\section{Zjets Background Modeling}\label{sec:dybk}
In this analysis, data-driven background modeling methods are used for majority of the backgrounds, including the Zjets background. The data-driven method in this analysis has benefits in that:
\begin{enumerate}
\item The number of MC events in the high energy region is limited, while data-driven method offers more statistics in our interesting region.
\item The reconstruction of ${p_{T}}^{miss}$ is more reliable in the data-driven background modeling methods, considering the detector contamination condition might vary between data and MC.
\end{enumerate}

As Figure~\ref{fig:mc_nvtx} and \ref{fig:mc_rho} shows, even with all the reweightings applied, the vertices number and $\rho$ (a parameter describing the contamination level of the detector) distributions still can not disagree nicely.
\begin{figure}[htbp!]
\centering
\includegraphics[width=0.46\linewidth, page=1]{figures/ReMiniSummer16_MC_GMCPhPtWt_tightzpt50_puWeightsummer16_metfilter_unblind_el_log_1pb.pdf}
\includegraphics[width=0.46\linewidth, page=1]{figures/ReMiniSummer16_MC_GMCPhPtWt_tightzpt50_puWeightsummer16_metfilter_unblind_mu_log_1pb.pdf}
\caption{N reco vertices distributions for electron (left) and muon (right) channels comparing data and MC.}
\label{fig:mc_nvtx}
\end{figure}

\begin{figure}[htbp!]
\centering
\includegraphics[width=0.46\linewidth, page=2]{figures/ReMiniSummer16_MC_GMCPhPtWt_tightzpt50_puWeightsummer16_metfilter_unblind_el_log_1pb.pdf}
\includegraphics[width=0.46\linewidth, page=2]{figures/ReMiniSummer16_MC_GMCPhPtWt_tightzpt50_puWeightsummer16_metfilter_unblind_mu_log_1pb.pdf}
\caption{$\rho$ distributions for electron (left) and muon (right) channels comparing data and MC.}
\label{fig:mc_rho}
\end{figure}

\vspace{0.3cm}
The Zjets background is characterized by a transversely boosted Z boson and a hadronic recoil balancing the momentum of the Z boson. The observed ${p_{T}}^{miss}$ in this background is completely instrumental. The photon$+$jets events have a similar kinematic signature as Zjets events, and can be used to model the Zjets background in a data-driven way.

\subsection{$\gamma$jets Event Selection}
The $\gamma$jets events are selected from the SinglePhoton dataset with HLT \texttt{HLT\_Photon``PT''\_R9Id90\_HE10\_IsoM}, for \texttt{PT} = 22, 30, 36, 50, 75, 90, 120,165 GeV. The \texttt{Loose} Photon ID defined recommended by the EGamma POG is applied. Furthermore, MET filters listed in Section~\ref{sec:metfilter} are also required in the photon data selection.

\vspace{0.3cm}
Even with the \texttt{Loose} Photon ID applied, the photons are still contaminated by many fake photons from sources such as ECAL APT sparks, ECAL noise that has not been flagged out, and beam halo photons. Therefore, additional selections are applied as listed below:
\begin{itemize}
\item Only one reconstructed photon in the event. 
%\item Additional fake photon events cleansing  $i\eta- i\phi$ ($ix-iy$) filter maps as shown on Figure~\ref{fig:gjets_photon_clean_eta_phi_map}. Notice that the entire region around $\phi=0$, and $\pi$ in the $x-z$ plain on the ECAL Endcap are flagged out, where beam halo photons are dominant.
\item Additional fake photon events cleansing filter applied based on problematic ECAL channel maps.
\item \texttt{sigmaIetaIeta}$>0.001$
\item \texttt{sigmaIphiIphi}$>0.001$  
\item ``Swiss Cross'':  $S = (1-E4/E1)<0.95$ , where E1 is the seed crystal energy, E4 is the sum of the energies in up, down, left and right crystals adjacent to the seed crystal.
\item ECAL seed crystal timing :   $t_0-1.5 ns < time < t_0+1.5 ns$, where $t_0$ is the peak time position. 
\item MIP Total Energy $< 4.9$ GeV to suppress halo induced showers in the ECAL.
\item Lepton veto: remove events with one or more reconstructed electron/muons with $ p_T > 10$ GeV, 
also remove events with jets contains more than 10 GeV of lepton energy (Charged EM energy, Muon Energy). 
This is to filter out processes such as $Z\to ee$ events, with one electron mis-identified as the photon. 
\end{itemize}

In addition, analogous to the Z boson preselection, $p_T ^{\gamma} > 50GeV$ is applied in the photon data selection.

\subsection{$\gamma$jets HLT Prescale Reweighting}
In the data taking of the CMS experiment, both the L1T and HLT can be prescaled in order to suppress the excessive low energy events. For the HLTs used in the $\gamma$jets event selection:
\texttt{HLT\_Photon<$P_T$>\_R9Id90\_HE10\_IsoM}, \\
 for $P_T$ = 22, 30, 36, 50, 75, 90, 120,165 GeV:
\begin{itemize}
\item $P_T$ threshold = 165 GeV: not pre-scaled, 
\item $P_T$ threshold = 50, 75, 90, and 120 GeV:  pre-scaled at only HLT.  
\item $P_T$ threshold = 22, 30, 36 GeV: pre-scaled at both L1T and HLT, 
\end{itemize}

Considering only photon with $p_T$ beyond 50 GeV would be selected, the L1T prescale has negligible effect on our selected data. The HLT prescale factor is obtained from the SinglePhoton data for each event, and applied as weight to cancel out the HLT prescale effect on the $p_T$ spectrum of the photons. Figure~\ref{fig:photon_pt_prescale} shows the photon $p_T$ spectrum with and without HLT prescale weight.


\begin{figure}[htbp]
\begin{center}
\includegraphics[width=0.86\linewidth]{figures/bg_photonHLT_reweight.pdf}
\caption{Photon $p_T$ distributions with and without HLT prescale weight. }
\label{fig:photon_pt_prescale}
\end{center}
\end{figure}

\subsection{Physical ${p_{T}}^{miss}$ Subtraction}
To better describe the kinematics of the $\gamma$jets events, events with physical ${p_{T}}^{miss}$ are expected to be subtracted from the SinglePhoton dataset. MC samples shown in Table~\ref{tab:bg_photonMC} are used to describe the single photon dataset. Figure~\ref{fig:pho_pteta}, \ref{fig:pho_met}, and \ref{fig:pho_metpara} shows the SinglePhoton data and MC agreement for 2016 full dataset. The SinglePhoton HLT set discussed above has been applied on both MC and data, and the HLT prescale reweighting applies too. A simple cutting-and-countting photon trigger efficiency is performed and the ratios between data and MC are applied as HLT efficiency SF on the MC samples. Nice agreement between the SinglePhoton data and MC is observed.

\begin{table}[htbh]
  \begin{center}
\begin{footnotesize}
    \caption{
      MC samples and their cross-sections for describing photon data and for physical ${p_{T}}^{miss}$ subtraction, Summer16 miniAODv2.
      \label{tab:bg_photonMC}}
    \begin{tabular}{l l}
      \hline
      MC Dataset & $\sigma [pb]$\\
      \hline\hline
  {\bf Instrumental ${p_{T}}^{miss}$ } & \\ \hline
       GJets\_HT-*To*\_TuneCUETP8M1\_13TeV-madgraphMLM-pythia8 & 32701  (LO) \\
       QCD\_Pt-*to*\_EMEnriched\_TuneCUETP8M1\_13TeV\_pythia8 & $1.86049\times 10^{7}$ (LO) \\
      \hline
      \hline
     {\bf Physical ${p_{T}}^{miss}$ } &\\ \hline
       DYJetsToLL\_M-50\_TuneCUETP8M1\_13TeV-amcatnloFXFX-pythia8 & $5765.4$  (NNLO)\\
       ZJetsToNuNu\_HT-*To*\_13TeV-madgraph & $457.081$  (NLO)\\
       WJetsToLNu\_HT-*To*\_TuneCUETP8M1\_13TeV-madgraphMLM-pythia8    & $2144.75$ (NLO) \\
       ZNuNuGJets\_MonoPhoton\_PtG-130\_TuneCUETP8M1\_13TeV-madgrap & $0.183\times1.43$ \\
       ZNuNuGJets\_MonoPhoton\_PtG-40to130\_TuneCUETP8M1\_13TeV-madgrap & $2.816\times1.43$ \\
       WGToLNuG\_TuneCUETP8M1\_13TeV-madgraphMLM-pythia8 & $585.8\times2.51$ \\
       TTGJets\_TuneCUETP8M1\_13TeV-amcatnloFXFX-madspin-pythia8 & 3.697 (NLO) \\
       ST\_t-channel\_top\_4f\_leptonDecays\_13TeV-powheg-pythia8\_TuneCUETP8M1 & 136.02 (NLO)\\
       ST\_t-channel\_antitop\_4f\_leptonDecays\_13TeV-powheg-pythia8\_TuneCUETP8M1 & 80.95 (NLO)\\
       ST\_tW\_top\_5f\_inclusiveDecays\_13TeV-powheg-pythia8\_TuneCUETP8M1 & 35.6  (NNLO)\\
       ST\_tW\_antitop\_5f\_inclusiveDecays\_13TeV-powheg-pythia8\_TuneCUETP8M1 & 35.6  (NNLO)\\
       TGJets\_TuneCUETP8M1\_13TeV\_amcatnlo\_madspin\_pythia8 & 2.967 (NLO)\\
      \hline\hline
    \end{tabular}
    \end{footnotesize}
  \end{center}
\end{table}


\begin{figure}[htbp!]
\centering
\includegraphics[width=0.48\linewidth, page=3]{figures/ReMiniAODSummer16HLT_FixXsec_SepProc_PhPtWt_tight_puWeightsummer16_unblind_log_.pdf}
\includegraphics[width=0.48\linewidth, page=4]{figures/ReMiniAODSummer16HLT_FixXsec_SepProc_PhPtWt_tight_puWeightsummer16_unblind_log_.pdf}
\caption{The original photon $p_T$ (left) and $\eta$ (right) distributions and the MC sample description for the SinglePhoton dataset. }
\label{fig:pho_pteta}
\end{figure}

\begin{figure}[htbp!]
\centering
\includegraphics[width=0.48\linewidth, page=6]{figures/ReMiniAODSummer16HLT_FixXsec_SepProc_PhPtWt_tight_puWeightsummer16_unblind_log_.pdf}
\includegraphics[width=0.48\linewidth, page=9]{figures/ReMiniAODSummer16HLT_FixXsec_SepProc_PhPtWt_tight_puWeightsummer16_unblind_log_.pdf}
\caption{The photon ${p_{T}}^{miss}$ (left) and $sumET$ (right) distributions and the MC sample description for the SinglePhoton dataset. }
\label{fig:pho_met}
\end{figure}

\begin{figure}[htbp!]
\centering
\includegraphics[width=0.48\linewidth, page=7]{figures/ReMiniAODSummer16HLT_FixXsec_SepProc_PhPtWt_tight_puWeightsummer16_unblind_log_.pdf}
\includegraphics[width=0.48\linewidth, page=8]{figures/ReMiniAODSummer16HLT_FixXsec_SepProc_PhPtWt_tight_puWeightsummer16_unblind_log_.pdf}
\caption{The photon ${p_{T}}^{miss}_\parallel$ (left) and ${p_{T}}^{miss}_\perp$ (right) distributions and the MC sample description for the SinglePhoton dataset. }
\label{fig:pho_metpara}
\end{figure}

\vspace{0.3cm}
For the purpose of physical ${p_{T}}^{miss}$ subtraction from the single photon data, the physical ${p_{T}}^{miss}$ MC samples are used together with the SinglePhoton dataset, but with a scale factor of $-1$.

\subsection{Photon $p_T$ to Z $p_T$ Reweighting}
In theory, the kinematic signature of the $\gamma$jets events is expected to be similar to Zjets events, especially when it comes to high energy region where the mass of Z bosons can be neglected. But it cannot be taken for granted in experiment, where detector effects exist, and more importantly, the lepton selectoins for the Z boson reconstruction in this analysis cannot be repeated on the photons. To address this issue, the photon $p_T$ distribution is reweighted to match that of the Z $p_T$. 

\vspace{0.3cm}
The $p_T$ spectrum of the Z bosons is obtained from the MC sample DYJetsToLL\_M-50\_TuneCUETP8M1\_13TeV-amcatnloFXFX-pythia8, with inclusive cross-section of 5765.4 pb ($\pm 1.7\%$ PDF uncertainty) calculated at NNLO from FEWZ 3.1~\cite{bg_fewz}, and differential cross-section reweighted to that measured from the 2015 CMS data~\cite{bg_2015zjetxsec} on the generator level. The standard preselection is applied to the MC samples, with all efficiency reweightings applied. Figure~\ref{fig:photon_pt_weight_el} and \ref{fig:photon_pt_weight_mu} shows the photon $p_T$ reweighting function for electron channel and muon channel separately. 

\begin{figure}[htbp]
\centering
  \includegraphics[width=0.9\linewidth]{figures/study_gjets_data_allcorV2_modify_el.pdf}
  \caption{Photon $p_T$ reweighting function for electron channel.
 The uncertainty bands includes uncertainties from 2015 CMS Zjets differential cross-section measurements, the statistical uncertainty from Zjets MC sample and $\gamma$jets data sample, and the lepton trigger, ID, ISO efficiency scale factors.}
  \label{fig:photon_pt_weight_el}
\end{figure}

\begin{figure}[htbp]
\centering
  \includegraphics[width=0.9\linewidth]{figures/study_gjets_data_allcorV2_modify_mu.pdf}
  \caption{Photon $p_T$ reweighting function for muon channel.
 The uncertainty bands includes uncertainties from 2015 CMS Zjets differential cross-section measurements, the statistical uncertainty from Zjets MC sample and $\gamma$jets data sample, and the lepton trigger, ID, ISO efficiency scale factors.}
  \label{fig:photon_pt_weight_mu}
\end{figure}

\subsection{Photon Mass Generation}
Considering the mass of the leptonic Z boson is used in the transverse mass calculation (Equation~\ref{eqn:intro_MT}\ref{eqn:intro_MTalt}), it is crucial for the photons in the $\gamma$jets events to have "mass". This is done by a simple random number generation. The Z boson mass distribution is modeled as a function of Z boson $p_T$. A 2D histogram regarding the mass and $p_T$ of the simulated Z bosons is made. Then a mass is randomely generated corresponding to the mass distribution and assigned to a given photon, with its $p_T$ value considered.

\vspace{0.3cm}
Figure~\ref{fig:mz_el_zjets_gjets} and \ref{fig:mz_mu_zjets_gjets}, compare the Z mass distributions of Zjets MC and the simulated Z mass for $\gamma$jets data events, for electron and muon channels separately.

\begin{figure}[htbp!]
\centering
\includegraphics[width=0.46\linewidth]{figures/MC2_Rc36p46DtReCalib_RhoWt_GMCEtaWt_tightzpt50_puWeightmoriondMC_metfilter_el_log_1pb.pdf}
\includegraphics[width=0.46\linewidth]{figures/GJets2_BkgSub_Rc36p46DtReCalib_NonReso_RhoWt_GMCEtaWt_tightzpt50_puWeightmoriondMC_muoneg_gjet_metfilter_el_log_1pb.pdf}
\caption{Z mass distributions for electron channel, comparing Zjets MC (left) and the simulated Z mass for $\gamma$jets data events (right).}
\label{fig:mz_el_zjets_gjets}
\end{figure}

\begin{figure}[htbp!]
\centering
\includegraphics[width=0.46\linewidth]{figures/MC2_Rc36p46DtReCalib_RhoWt_GMCEtaWt_tightzpt50_puWeightmoriondMC_metfilter_mu_log_1pb.pdf}
\includegraphics[width=0.46\linewidth]{figures/GJets2_BkgSub_Rc36p46DtReCalib_NonReso_RhoWt_GMCEtaWt_tightzpt50_puWeightmoriondMC_muoneg_gjet_metfilter_mu_log_1pb.pdf}
\caption{Z mass distributions for muon channel, comparing Zjets MC (left) and the simulated Z mass for $\gamma$jets data events (right).}
\label{fig:mz_mu_zjets_gjets}
\end{figure}

\subsection{${p_{T}}^{miss}$ Hadronic Recoil Tuning}
A simple single-Gaussian based hadronic recoil fit is developed to tune the ${p_{T}}^{miss}$ of the $\gamma$jets data to match the ${p_{T}}^{miss}$ in the Zjets data.
This is needed because the energy scale resolutions of leptons, jets, photons are potentially different between $\gamma$jets data and Zjets data. Those differences may introduce difference in the ${p_{T}}^{miss}$ resolution and scale. 

\vspace{0.3cm}
The hadronic recoil fit is performed based on the specific Zjets kinematics that the Z boson and hadronic recoil balancing with each other in the transverse plain.
As a result, the resulting ${p_{T}}^{miss}$ vector should be pure instrumental, i.e. a 2-dimensional Gaussian function peaked at point $(x,y)=(0,0)$. However, the jets, leptons and photons are potentially have their energy underestimated or overestimated, such as the neutron hadron components in jets, even with the PF algorithms one cannot find a 
track to replace the less precisely measured hadronic energy in HCAL. These effects result in an imbalance in the Z $p_T$ boost direction, causing the ${p_{T}}^{miss}$ component parallel
to the Z $p_T$ direction having larger and asymmetric non-Gaussian tails on either the positive or negative side. 

\vspace{0.3cm}
Though the simple Gaussian based hadronic recoil fit will not fix this imbalance in the Z $p_T$ direction, it can at least be used to correct the Gaussian-core part of the ${p_{T}}^{miss}$ of the $\gamma$jets data to match the Zjets data. Gaussion fits are applied to describe the Gaussion-core of the ${p_{T}}^{miss}_\parallel$ and ${p_{T}}^{miss}_\perp$ for each Z $p_T$ or photon $p_T$ bin. Figure~\ref{fig:recoilfit_example_data} give some example plots showing the Gaussian fits for the $p_T$ bins 50-60 GeV for Zjets data, for muon channel and electron channel, for ${p_{T}}^{miss}_\parallel$ and ${p_{T}}^{miss}_\perp$, respectively.
\begin{figure}[htbp]
\begin{center}
\includegraphics[width=0.46\linewidth, page=21]{figures/SingleEMU_Run2016Full_03Feb2017_allcorV2_met_para_study_ZSelecLowLPt_mu.pdf}
\includegraphics[width=0.46\linewidth, page=56]{figures/SingleEMU_Run2016Full_03Feb2017_allcorV2_met_para_study_ZSelecLowLPt_mu.pdf}
\includegraphics[width=0.46\linewidth, page=21]{figures/SingleEMU_Run2016Full_03Feb2017_allcorV2_met_para_study_ZSelecLowLPt_el.pdf}
\includegraphics[width=0.46\linewidth, page=21]{figures/SingleEMU_Run2016Full_03Feb2017_allcorV2_met_para_study_ZSelecLowLPt_el.pdf}
\includegraphics[width=0.46\linewidth, page=56]{figures/SingleEMU_Run2016Full_03Feb2017_allcorV2_met_para_study_ZSelecLowLPt_el.pdf}
\caption{Example plots for single Gaussian-based ${p_{T}}^{miss}$ hadronic recoil fit of selected Z $p_T$ bin for Zjets data, muon channel (upper), electron channel (lower),${p_{T}}^{miss}_\parallel$ (left), ${p_{T}}^{miss}_\perp$ (right).}
\label{fig:recoilfit_example_data}
\end{center}
\end{figure}


\vspace{0.3cm}
The comparison of the recoil fit results between Zjets data and $\gamma$jets data are shown on 
Figure~\ref{fig:recoilfit_met_peak_reso_compare_data_gjets_mu}
and \ref{fig:recoilfit_met_peak_reso_compare_data_gjets_el}
for muon channel and electron channel respectively. The $\gamma$jets data ${p_{T}}^{miss}_\parallel$ peak and the resolution of ${p_{T}}^{miss}_\parallel$ and ${p_{T}}^{miss}_\perp$ are then corrected to match the Zjets data respectively for muon and electron channels based on Figure~\ref{fig:recoilfit_met_peak_reso_compare_data_gjets_mu} and \ref{fig:recoilfit_met_peak_reso_compare_data_gjets_el}. The correction for the photon data is done in the following way:
\begin{itemize}
\item Shift the ${p_{T}}^{miss}_\parallel$ peak positions of $\gamma$jets data to match that of the Zjets data;
\item Scale the ${p_{T}}^{miss}_\parallel$ and ${p_{T}}^{miss}_\perp$ resolution of $\gamma$jets data to match that of the Zjets data.
\end{itemize}

\begin{figure}[htbp]
\begin{center}
\includegraphics[width=0.46\linewidth, page=1]{figures/plots_SingleEMU_Run2016Full_03Feb2017_allcorV2_met_para_study_ZSelecLowLPt_mu_VS_SinglePhoton_Run2016Full_03Feb2017_allcorV2_NoRecoil_met_para_study_ZSelecLowLPt_mu.pdf}
\includegraphics[width=0.46\linewidth, page=5]{figures/plots_SingleEMU_Run2016Full_03Feb2017_allcorV2_met_para_study_ZSelecLowLPt_mu_VS_SinglePhoton_Run2016Full_03Feb2017_allcorV2_NoRecoil_met_para_study_ZSelecLowLPt_mu.pdf}
\includegraphics[width=0.46\linewidth, page=3]{figures/plots_SingleEMU_Run2016Full_03Feb2017_allcorV2_met_para_study_ZSelecLowLPt_mu_VS_SinglePhoton_Run2016Full_03Feb2017_allcorV2_NoRecoil_met_para_study_ZSelecLowLPt_mu.pdf}
\includegraphics[width=0.46\linewidth, page=7]{figures/plots_SingleEMU_Run2016Full_03Feb2017_allcorV2_met_para_study_ZSelecLowLPt_mu_VS_SinglePhoton_Run2016Full_03Feb2017_allcorV2_NoRecoil_met_para_study_ZSelecLowLPt_mu.pdf}
\caption{Comparison of the recoil fitted peak positions and Gaussian resolutions for ${p_{T}}^{miss}_\parallel$ and ${p_{T}}^{miss}_\perp$ between di-lepton data and $\gamma$jets data for the muon channel. Upper two for 
${p_{T}}^{miss}_\parallel$, bottom two for ${p_{T}}^{miss}_\perp$.}
\label{fig:recoilfit_met_peak_reso_compare_data_gjets_mu}
\end{center}
\end{figure}

\begin{figure}[htbp]
\begin{center}
\includegraphics[width=0.46\linewidth, page=1]{figures/plots_SingleEMU_Run2016Full_03Feb2017_allcorV2_met_para_study_ZSelecLowLPt_el_VS_SinglePhoton_Run2016Full_03Feb2017_allcorV2_NoRecoil_met_para_study_ZSelecLowLPt_el.pdf}
\includegraphics[width=0.46\linewidth, page=5]{figures/plots_SingleEMU_Run2016Full_03Feb2017_allcorV2_met_para_study_ZSelecLowLPt_el_VS_SinglePhoton_Run2016Full_03Feb2017_allcorV2_NoRecoil_met_para_study_ZSelecLowLPt_el.pdf}
\includegraphics[width=0.46\linewidth, page=3]{figures/plots_SingleEMU_Run2016Full_03Feb2017_allcorV2_met_para_study_ZSelecLowLPt_el_VS_SinglePhoton_Run2016Full_03Feb2017_allcorV2_NoRecoil_met_para_study_ZSelecLowLPt_el.pdf}
\includegraphics[width=0.46\linewidth, page=7]{figures/plots_SingleEMU_Run2016Full_03Feb2017_allcorV2_met_para_study_ZSelecLowLPt_el_VS_SinglePhoton_Run2016Full_03Feb2017_allcorV2_NoRecoil_met_para_study_ZSelecLowLPt_el.pdf}
\caption{Comparison of the recoil fitted peak positions and Gaussian resolutions for ${p_{T}}^{miss}_\parallel$ and ${p_{T}}^{miss}_\perp$ between di-lepton data and $\gamma$jets data for the electron channel. Upper two for 
${p_{T}}^{miss}_\parallel$, bottom two for ${p_{T}}^{miss}_\perp$.}
\label{fig:recoilfit_met_peak_reso_compare_data_gjets_el}
\end{center}
\end{figure}

\clearpage
\section{Non-resonant Background Modeling}
In this analysis the non-resonance background containing mostly $t\bar{t}$ and WW events is one of the main background, especially in the signal region, because ZJets events have no physical ${p_{T}}^{miss}$ in their final status and therefore can be suppressed by selecting events with high ${p_{T}}^{miss}$. 

\vspace{0.3cm}
A data driven method is used in the analysis for the non-resonance background modeling. The method is to use the $e\mu$ pairs to describe the non-resonance background in ll (ee or $\mu \mu$) event, based on the fact that as for the non-resonance background, $e\mu$ pairs are supposed to have the same kinematic behavior and similar cross-section as ll (ee or $\mu \mu$) does.

\subsection{$e\mu$ Pair Selection}
The 35.8 fb$^{-1}$ 2016 MuonEG dataset is used for the $e\mu$ pair selection.  Events with one $e\mu$ pair or more (no HLT requirement) are selected. If more than 1 $e\mu$ pair exit in one event, the pair with invariant mass closest to Z mass is selected (analogy to the data selection). Electrons are required to pass Loose ID with pfIso, and muons are required to pass High $p_T$ ID and tracker isolation. 

\vspace{0.3cm}
When the $e\mu$ sample is used as background in the electron channel, selections below are applied (analogy to the electron selection in data: 
\begin{itemize}
\item Leading lepton $p_{T}>120GeV$, $|\eta|<2.5$, 
\item Subleading lepton $p_{T}>35GeV$, $|\eta|<2.5$
\end{itemize}
When the sample is used in the muon channel, selections are (analogy to the muon selection in data): 
\begin{itemize}
\item Leading lepton $p_{T}>55GeV$, $|\eta|<2.4$
\item Subleading lepton $p_{T}>20GeV$, $|\eta|<2.4$. 
\end{itemize}

\subsection{$e\mu$ Pair Event-based Reweighting}
Based on the assuption that electrons and muons have the same behavior in the non-resonant background, the kinematic variables' distributions of the electron and muon in the $e\mu$ pair are expected to be identical. However it is not exactly true. Figure~\ref{fig:nonresmuelpt} shows the $p_T$ distribution of electron and muon, and their ratio in the MuonEG dataset.

\begin{figure}[htbp]
\begin{center}
\includegraphics[width=0.95\linewidth]{figures/nonresmuelpt.pdf}
\caption{electron and muon $p_T$ distribution (left) and ratio (right) in the $e\mu$ pair sample selected from MuonEG dataset.}
\label{fig:nonresmuelpt}
\end{center}
\end{figure}

\vspace{0.3cm}
The discrepancy between the $p_T$ distributions of electrons and muons can result from a lot of factors, such as detector effects, HLT/ID/Iso efficiencies. Event based weighting factors are calculated according to the ratio plot in Figure~\ref{fig:nonresmuelpt}(right):
\begin{itemize}
\item  When modeling the electron channel, the weighting factor of the event is the content of the bin corresponding to the true muon in that event. 
\item  When modeling the muon channel, the weighting factor of the event is the inverse of the content of the bin corresponding to the true electron in the event. 
\end{itemize}
Figure~\ref{fig:nonresmuelptel} and Figure~\ref{fig:nonresmuelptmu} shows the electron and muon $p_T$ distribution after the reweighting, in muon channel and in electron channel, as well as the ratio.

\begin{figure}[htbp]
\begin{center}
\includegraphics[width=0.95\linewidth]{figures/nonresmuelptel.pdf}
\caption{electron and muon $p_T$ distribution (left) and ratio (right) in the $e\mu$ pair sample (from MuonEG dataset), after the event based reweighting, in electron channel}
\label{fig:nonresmuelptel}
\end{center}
\end{figure}

\begin{figure}[htbp]
\begin{center}
\includegraphics[width=0.72\linewidth]{figures/nonresmuelptmu.pdf}
\caption{electron and muon $p_T$ distribution(left) and ratio(right) in the $e\mu$ pair sample (from MuonEG dataset), after the event based reweighting, in muon channel}
\label{fig:nonresmuelptmu}
\end{center}
\end{figure}

\vspace{0.3cm}
The agreement between the $p_T$ distributions of electrons and muons are improved with the event based reweighting for both the electron channel and muon channel. In the electron channel, both leptons behave like the electron before weighting, and in the muon channel both leptons behave like the muon before weighting. This reweighting suppresses the systematic uncertainty caused by the performance discrepancy between electrons and muons due to detector and reconstruction effects. With the reweighting there are still ~2\% of disagreement between the $p_T$ distributions of electrons and muons, and it is quoted in the systematic uncertainty of this method.

\subsection{HLT Efficiency Reweighting}
As HLT is required in our $\ell \ell$ data selection, while no such HLT for the $e\mu$ sample, data trigger efficiency is calculated and applied in the $e\mu$ sample to simulate the single lepton HLT. With the event based reweighting described above, the electrons and muons in the $e\mu$ pair sample behave very similarly and can be regarded as $\ell \ell$ pairs. Both SingleElectron and SingleMuon trigger efficiency for data are calculated for each event in the $e\mu$ pair data, corresponding to the leading lepton's $p_T$ and $\eta$, regardless whether the leading lepton is a muon or an electron. The trigger efficiency for data is described in Section~\ref{sec:bkg_trig}. The SingleElectron trigger efficiency is applied when the sample is used in the electron channel, similarly the SingleMuon trigger efficiency is applied when the sample is used in the muon channel. The uncertainty of the trigger efficiency is evaluated and quoted as the mean deviation (in percent) among all the selected events of each channel (see Table~\ref{tab:nonresuncert}).

\vspace{0.3cm}
When applying the SingleLepton trigger efficiency on the MuonEG dataset to emulate the data HLT, the MuonEG dataset is expected to have full acceptance in our preselection region. This explains why no HLT is required in the MuonEG data selection process. By not requiring any MuonEG HLT, events passing any MuonEG HLTs are accepted. Considering the high $p_T$ threshold in the lepton selection, the acceptance of the dataset is high enough and this effect can be ignored compared to the single lepton trigger efficiency uncertainty. In fact, the acceptance of the MuonEG dataset has negligible effects as long as it is consistent in our preselection region, as the $e\mu$ sample would be rescaled as discussed below. To clarify, the $p_T$ thresholds of the MuonEG dataset would not affact our event selection. For the muon channel, we have leading lepton $pt>60 GeV$, and subleading lepton $pt>20 GeV$. And the electron channel has higher lepton $p_T$ selection. The MuonEG dataset contains non-prescaled triggers:
\begin{itemize}
\item \texttt{HLT\_Mu23\_TrkIsoVVL\_Ele12\_CaloIdL\_TrackIdL\_IsoVL(\_DZ)} 
\item \texttt{HLT\_Mu12\_TrkIsoVVL\_Ele23\_CaloIdL\_TrackIdL\_IsoVL(\_DZ)}
\end{itemize}
and these $p_T$ threshold is much lower than our $p_T$ selecton.

\subsection{$e\mu$ Events Rescaling}
To rescale the $e\mu$ samples to fit the nonresonance background and study the agreement of the background sample and data, a nonresonant control region is defined. The scale factor is calculated by equation~\ref{eq:nonresscale}.
\begin{equation} \label{eq:nonresscale}
  Scale  =  \frac{N_{data,ll}-N_{MC,reson}}{N_{data,e\mu}}
\end{equation}

The $N_{data,ll}$ is the number of data events in the control region; $N_{MC,reson}$ is the number of SM background Z event in the control region, including Zjets background and resonance background; the $N_{data,e\mu}$ is the number of $e\mu$ pair data in the control region. To define the control region, the main idea is to suppress the standard deviation of the calculated scale factor, the standard deviation is calculated based on equation~\ref{eq:nonresscale}, and given in equation~\ref{eq:nonresscaledev}.
\begin{equation} \label{eq:nonresscaledev}
  \sigma_{Scale}  = \sqrt{\frac{\sigma^{2}_{N_{data,ll}}+\sigma^{2}_{N_{MC,reson}}}{(N_{data,ll}-N_{MC,reson})^{2}}+\frac{\sigma^{2}_{N_{data,e\mu}}}{N^{2}_{data,e\mu}}}\times \frac{N_{data,ll}-N_{MC,reson}}{N_{data,e\mu}}
\end{equation}

Here the statistic uncertanity is considered for both MC and data, and the PDF uncertainty is considered for resonant MC (1.5\%).

\vspace{0.3cm}
To suppress scale deviation and resonance background in the CR, selections below are applied: 
\begin{enumerate}
\item Z mass veto: invariant mass $M<70GeV$ or $M>110GeV$; 
\item $p_{T}^{Z} > X GeV$;
\end{enumerate}

To determine the X value ($p_{T}^{Z}$ cut level), a scan over various X values is done, scale and uncertainty vs X is shown in Table~\ref{tab:nonressfdev}. X=60 gives relatively small uncertainty, and is used to define the control region. The scale uncertainty corresponding to X=60 is also counted as the systematic uncertainty.

\begin{table}[htbp]
  \begin{center}
    \caption{
      scale and deviation vs X($p_{T}^{Z}$ cut value) scan result.      
      \label{tab:nonressfdev}}
    \begin{tabular}{c c c}
      \hline\hline
      X & electron channel scale & muon channel scale\\
      \hline
      0  & 0.3442$\pm$0.0140 & 0.8215$\pm$0.0519 \\
      10 & 0.3442$\pm$0.0140 & 0.7762$\pm$0.0388 \\
      20 & 0.3442$\pm$0.0140 & 0.7364$\pm$0.0290 \\
      30 & 0.3442$\pm$0.0140 & 0.7046$\pm$0.0238 \\
      40 & 0.3442$\pm$0.0140 & 0.6962$\pm$0.0202 \\
      50 & 0.3443$\pm$0.0140 & 0.6895$\pm$0.0182 \\
      60 & 0.3463$\pm$0.0140 & 0.6871$\pm$0.0171 \\
      70 & 0.3490$\pm$0.0142 & 0.6907$\pm$0.0167 \\
      80 & 0.3560$\pm$0.0146 & 0.6952$\pm$0.0170 \\
      90 & 0.3635$\pm$0.0156 & 0.7092$\pm$0.0182 \\
      100 & 0.3733$\pm$0.0169 & 0.7249$\pm$0.0205 \\
      110 & 0.3771$\pm$0.0186 & 0.7402$\pm$0.0234 \\
      \hline\hline
    \end{tabular}
  \end{center}
\end{table}

\vspace{0.3cm}
Electron channel plots in the control region are shown in the Figure~\ref{fig:nonreselcr} and those for the muon channel are shown in the Figure~\ref{fig:nonresmucr}.

\begin{figure}[htbp]
\begin{center}
\includegraphics[width=0.39\linewidth, page=1]{figures/test_metzpt50_RhoWt_puWeight68075_metfilter_el_.pdf}
\includegraphics[width=0.39\linewidth, page=2]{figures/test_metzpt50_RhoWt_puWeight68075_metfilter_el_.pdf}
\includegraphics[width=0.39\linewidth, page=3]{figures/test_metzpt50_RhoWt_puWeight68075_metfilter_el_.pdf}
\includegraphics[width=0.39\linewidth, page=4]{figures/test_metzpt50_RhoWt_puWeight68075_metfilter_el_.pdf}
\caption{data-driven non-resonance background in the control region, electron channel plots}
\label{fig:nonreselcr}
\end{center}
\end{figure}

\begin{figure}[htbp]
\begin{center}
\includegraphics[width=0.39\linewidth, page=1]{figures/test_metzpt50_RhoWt_puWeight68075_metfilter_mu_.pdf}
\includegraphics[width=0.39\linewidth, page=2]{figures/test_metzpt50_RhoWt_puWeight68075_metfilter_mu_.pdf}
\includegraphics[width=0.39\linewidth, page=3]{figures/test_metzpt50_RhoWt_puWeight68075_metfilter_mu_.pdf}
\includegraphics[width=0.39\linewidth, page=4]{figures/test_metzpt50_RhoWt_puWeight68075_metfilter_mu_.pdf}
\caption{data-driven non-resonance background in the control region, muon channel plots}
\label{fig:nonresmucr}
\end{center}
\end{figure}

\vspace{0.3cm}
The control region plots shows that resonance background is heavily suppressed in the CR and data driven non-resonance background agrees well with the data. Also NonReson $e\mu$ should have similar cross section with NonReson to ll, which means that the sum of the scale in electron channel and the scale in muon channel should be close to 1, and from Table~\ref{tab:nonressfdev} we see that at X=60, Scale(ee channel)=0.346, Scale($\mu\mu$ channel)=0.687, adding up to 1.033, which agrees with the prediction.

\subsection{Comparing to MC Modeling}
Figure~\ref{fig:nonreselzmasscr100} is a comparison of the nonresonant background modeled by data-driven method and the $t\bar{t}+WW$ MC samples as a cross-check of the data-driven method, with $70GeV<M_{Z}<110GeV$ selection, ${p_{T}}^{miss}>50GeV$ and $p_{T}^{Z}>100GeV$ selection (standard signal region selections in this analysis), for electron channel. And similarly Figure~\ref{fig:nonresmuzmasscr100} is for muon region. The data-driven modeling method generally gives a very similar non-resonant background distribution as the MC. However, in the electron channel the yield of the data-driven method is slightly higher than the MC modeling.

\begin{figure}[htbp]
\begin{center}
\includegraphics[width=0.39\linewidth, page=1]{figures/test_zpt100_log_RhoWt_puWeight68075_metfilter_el_.pdf}
\includegraphics[width=0.39\linewidth, page=3]{figures/test_zpt100_log_RhoWt_puWeight68075_metfilter_el_.pdf}
\includegraphics[width=0.39\linewidth, page=6]{figures/test_zpt100_log_RhoWt_puWeight68075_metfilter_el_.pdf}
\includegraphics[width=0.39\linewidth, page=2]{figures/test_zpt100_log_RhoWt_puWeight68075_metfilter_el_.pdf}
\includegraphics[width=0.39\linewidth, page=5]{figures/test_zpt100_log_RhoWt_puWeight68075_metfilter_el_.pdf}
\includegraphics[width=0.39\linewidth, page=4]{figures/test_zpt100_log_RhoWt_puWeight68075_metfilter_el_.pdf}
\caption{data-driven non-resonance background and MC non-resonance background comparison, with Z mass selection and $p_{T}^{Z}>100GeV, {p_{T}}^{miss}>50GeV$, electron channel plots}
\label{fig:nonreselzmasscr100}
\end{center}
\end{figure}

\begin{figure}[htbp]
\begin{center}
\includegraphics[width=0.39\linewidth, page=1]{figures/test_zpt100_log_RhoWt_puWeight68075_metfilter_mu_.pdf}
\includegraphics[width=0.39\linewidth, page=3]{figures/test_zpt100_log_RhoWt_puWeight68075_metfilter_mu_.pdf}
\includegraphics[width=0.39\linewidth, page=6]{figures/test_zpt100_log_RhoWt_puWeight68075_metfilter_mu_.pdf}
\includegraphics[width=0.39\linewidth, page=2]{figures/test_zpt100_log_RhoWt_puWeight68075_metfilter_mu_.pdf}
\includegraphics[width=0.39\linewidth, page=5]{figures/test_zpt100_log_RhoWt_puWeight68075_metfilter_mu_.pdf}
\includegraphics[width=0.39\linewidth, page=4]{figures/test_zpt100_log_RhoWt_puWeight68075_metfilter_mu_.pdf}
\caption{data-driven non-resonance background and MC non-resonance background comparison, with Z mass selection and $p_{T}^{Z}>100GeV, {p_{T}}^{miss}>50GeV$,muon channel plots}
\label{fig:nonresmuzmasscr100}
\end{center}
\end{figure}

\vspace{0.3cm}
The yield discrepancy is evaluated and half of the discrepancy value is quoted as systematic uncertainty (6.7\% for electron channel).

\subsection{Uncertainty Table}
In addition to the systematic uncertainty of this data-driven method discussed above, another potential source is that all muons have $|\eta|<2.4$, while in the electron channel, each lepton from real non-resonance background can have $|\eta|$ between 2.4 and 2.5, which differs from the $e\mu$ pair data-driven background. Fortunately, less than 0.1\% events with $p_{T}^{Z}>100GeV, {p_{T}}^{miss}>50GeV$ in the electron channel have either one electron with $|\eta|$ between 2.4 and 2.5, which means that the effect can be neglected. 

\vspace{0.3cm}
Table~\ref{tab:nonresuncert} summarized the uncertainties that have been discussed concerning this data-driven nonresonant modeling method.
\begin{table}[htbp]
\begin{small}
  \begin{center}
    \caption{
      data-driven nonresonant modeling method uncertainties
      \label{tab:nonresuncert}}
    \begin{tabular}{c|c c}
      \hline\hline
      Uncertainty & electron channel & muon channel \\
      \hline
      $\mu$/e Detector-wise/reconstruction difference & 2\% & 2\% \\
      Trigger efficiency & 6.0\% & 1.3\% \\
      Stat. uncert. of Data and MC, PDF/QCD uncert. of subtracted MCs  & 4.0\% & 2.4\% \\
      Data-driven vs. MC disagreement & 6.7\% & 0.1\% \\
      \hline
      Total  & 10.0\% & 3.4\% \\
      \hline\hline
    \end{tabular}
  \end{center}
\end{small}
\end{table}

\clearpage
\section{Resonant Background Modeling}
The resonant background in this analysis contains mainly SM $qq\rightarrow ZZ\rightarrow 2\ell 2\nu$ process, as well as WZ processes and ZZ processes with llqq or 4l final states. This fraction of background is modeled by MC samples. For the $qq\rightarrow ZZ\rightarrow 2\ell 2\nu$ sample (ZZTo2L2Nu\_13TeV\_powheg\_pythia8), NNLO QCD correction~\cite{bg_nnloqcd} and NLO EW correction~\cite{bg_nloqed1,bg_nloqed2} are applied.

\vspace{0.3cm}
The NNLO/NLO QCD correction is parametrized and applied as a function of M(ZZ) 
at generator level. 
The correction and the uncertainty band are shown on Figure~\ref{fig:qqzz_nnlo_qcd}.
The average NNLO/NLO QCD correction k-factor is 1.11 with an uncertainty of 3\%.
For generator level M(ZZ)~$>$~500~GeV, the average k-factor and uncertainty are applied. 

\begin{figure}[htbp!]
\centering
  \includegraphics[width=0.48\linewidth]{figures/h_nnlo_to_nlo_vs_mzz.pdf}
  \caption{The NNLO/NLO QCD correction and error band as a function of generator level M(ZZ) for SM qqZZ process.}
  \label{fig:qqzz_nnlo_qcd}
\end{figure}

\vspace{0.3cm}
The NLO/LO EW correction is parametrized as a function of initial states quark flavors and event kinematic variables $\hat{s}$ and $\hat{t}$
in the center of mass frame at generator level. 
Variable $\hat{s}$ is the center of mass energy, corresponding to $m_{ZZ}$,
and $\hat{t}$ is computed as
\begin{align*}
\hat{t} = \left(p^*_{q_1}-p^*_{Z_1}\right)^2 & = p_{q_1}^{*2} +
p_{Z_1}^{*2} - 2 p^*_{q_1} \cdot p^*_{Z_1} \\
& \simeq 0 + m_{Z}^{2} - 2 \left( \frac{\hat{s}}{4} -
\frac{\sqrt{\hat{s}}}{2} \cos{\theta} \sqrt{\frac{\hat{s}}{4} -
m_{Z}^{2}} \right) \\
& = m_{Z}^{2} - \frac{\hat{s}}{2} + \cos{\theta}
\sqrt{\frac{\hat{s}^2}{4} - m_{Z}^{2}\hat{s}},
\end{align*}
where $p^*_{q_1}$ is the four momentum of the first quark initiating the hard process and 
$p^*_{Z_1}$  is the four momentum of the first Z boson, and quarks masses are neglected.
The angle $\theta$ is the angle between the Z boson and the direction of the
incident quarks in the center-of-mass frame of the two Z bosons. 
Considering the radiation of gluons emitted at small angles, the direction of the incident
quarks is approximately computed as 
\begin{equation*}
\cos{\theta} = \frac{\hat{\vec{p}}_{q_1b} -
\hat{\vec{p}}_{q_2b}}{\left|\left( \hat{\vec{p}}_{q_1b} -
\hat{\vec{p}}_{q_2b} \right)\right|} \cdot \hat{\vec{p}}_{Z_1b},
\end{equation*}
where $\hat{\vec{p}}_{q_i/Z_ib}$ represents the unitary vector of the
$i$th quark/$Z$ boson after the Lorentz boost. 

\vspace{0.3cm}
The average k-factor for NLO/LO EW correction is 0.95 with an uncertainty of 3~\%. 
The correction function and the uncertainty band is shown on Figure~\ref{fig:qqzz_nlo_ew}
as a function of generator level $m_{ZZ}$. 
The NLO/LO EW correction is only appropriate for on-shell Z with $m_{ZZ}>2 m_{Z}$. 

\begin{figure}[htbp!]
\centering
  \includegraphics[width=0.48\linewidth]{figures/ewkfactor.pdf}
  \caption{The NLO/LO EW correction and error band as a function of generator level M(ZZ) for SM qqZZ process.}
  \label{fig:qqzz_nlo_ew}
\end{figure}

\clearpage
\section{Preselection Plots}
Figure~\ref{fig:gjet_rho} to \ref{fig:gjet_metperp} are the data vs background plots in the preselection region, with all the backgrounds modeled in the method discussed above. 

\begin{figure}[htbp!]
\centering
\includegraphics[width=0.46\linewidth, page=1]{figures/ReMiniSummer16_DT_PhReMiniMCRcFixXsec_GMCPhPtWt_tightzpt50_puWeightsummer16_muoneg_gjet_metfilter_unblind_el_log_1pb.pdf}
\includegraphics[width=0.46\linewidth, page=1]{figures/ReMiniSummer16_DT_PhReMiniMCRcFixXsec_GMCPhPtWt_tightzpt50_puWeightsummer16_muoneg_gjet_metfilter_unblind_mu_log_1pb.pdf}
\caption{N distributions for electron (left) and muon (right) channels
comparing data and background.}
\label{fig:gjet_nvtx}
\end{figure}

\begin{figure}[htbp!]
\centering
\includegraphics[width=0.46\linewidth, page=2]{figures/ReMiniSummer16_DT_PhReMiniMCRcFixXsec_GMCPhPtWt_tightzpt50_puWeightsummer16_muoneg_gjet_metfilter_unblind_el_log_1pb.pdf}
\includegraphics[width=0.46\linewidth, page=2]{figures/ReMiniSummer16_DT_PhReMiniMCRcFixXsec_GMCPhPtWt_tightzpt50_puWeightsummer16_muoneg_gjet_metfilter_unblind_mu_log_1pb.pdf}
\caption{$\rho$ reco vertices distributions for electron (left) and muon (right)
channels comparing data and background.}
\label{fig:gjet_rho}
\end{figure}

\begin{figure}[htbp!]
\centering
\includegraphics[width=0.46\linewidth,page=3]{figures/ReMiniSummer16_DT_PhReMiniMCRcFixXsec_GMCPhPtWt_tightzpt50_puWeightsummer16_muoneg_gjet_metfilter_unblind_el_log_1pb.pdf}
\includegraphics[width=0.46\linewidth,page=3]{figures/ReMiniSummer16_DT_PhReMiniMCRcFixXsec_GMCPhPtWt_tightzpt50_puWeightsummer16_muoneg_gjet_metfilter_unblind_mu_log_1pb.pdf}
\caption{$m_T$ distributions for electron (left) and muon (right) channels
comparing data and background,
wide mass window.}
\label{fig:gjet_mt_wide}
\end{figure}

\begin{figure}[htbp!]
\centering
\includegraphics[width=0.46\linewidth,page=5]{figures/ReMiniSummer16_DT_PhReMiniMCRcFixXsec_GMCPhPtWt_tightzpt50_puWeightsummer16_muoneg_gjet_metfilter_unblind_el_log_1pb.pdf}
\includegraphics[width=0.46\linewidth,page=5]{figures/ReMiniSummer16_DT_PhReMiniMCRcFixXsec_GMCPhPtWt_tightzpt50_puWeightsummer16_muoneg_gjet_metfilter_unblind_mu_log_1pb.pdf}
\caption{$m_T$ distributions for electron (left) and muon (right) channels
comparing data and background,
narrow mass window.}
\label{fig:gjet_mt_narrow}
\end{figure}

\begin{figure}[htbp!]
\centering
\includegraphics[width=0.46\linewidth,page=8]{figures/ReMiniSummer16_DT_PhReMiniMCRcFixXsec_GMCPhPtWt_tightzpt50_puWeightsummer16_muoneg_gjet_metfilter_unblind_el_log_1pb.pdf}
\includegraphics[width=0.46\linewidth,page=8]{figures/ReMiniSummer16_DT_PhReMiniMCRcFixXsec_GMCPhPtWt_tightzpt50_puWeightsummer16_muoneg_gjet_metfilter_unblind_mu_log_1pb.pdf}
\caption{Z mass distributions for electron (left) and muon (right) channels
comparing data and background.}
\label{fig:gjet_mz}
\end{figure}

\begin{figure}[htbp!]
\centering
\includegraphics[width=0.46\linewidth,page=9]{figures/ReMiniSummer16_DT_PhReMiniMCRcFixXsec_GMCPhPtWt_tightzpt50_puWeightsummer16_muoneg_gjet_metfilter_unblind_el_log_1pb.pdf}
\includegraphics[width=0.46\linewidth,page=9]{figures/ReMiniSummer16_DT_PhReMiniMCRcFixXsec_GMCPhPtWt_tightzpt50_puWeightsummer16_muoneg_gjet_metfilter_unblind_mu_log_1pb.pdf}
\caption{Z $p_T$ distributions for electron (left) and muon (right) channels
comparing data and background,
wide binning.}
\label{fig:gjet_zpt_wide}
\end{figure}

\begin{figure}[htbp!]
\centering
\includegraphics[width=0.46\linewidth,page=10]{figures/ReMiniSummer16_DT_PhReMiniMCRcFixXsec_GMCPhPtWt_tightzpt50_puWeightsummer16_muoneg_gjet_metfilter_unblind_el_log_1pb.pdf}
\includegraphics[width=0.46\linewidth,page=10]{figures/ReMiniSummer16_DT_PhReMiniMCRcFixXsec_GMCPhPtWt_tightzpt50_puWeightsummer16_muoneg_gjet_metfilter_unblind_mu_log_1pb.pdf}
\caption{Z $p_T$ distributions for electron (left) and muon (right) channels
comparing data and background,
narrow binning.}
\label{fig:gjet_zpt_narrow}
\end{figure}

\begin{figure}[htbp!]
\centering
\includegraphics[width=0.46\linewidth,page=16]{figures/ReMiniSummer16_DT_PhReMiniMCRcFixXsec_GMCPhPtWt_tightzpt50_puWeightsummer16_muoneg_gjet_metfilter_unblind_el_log_1pb.pdf}
\includegraphics[width=0.46\linewidth,page=16]{figures/ReMiniSummer16_DT_PhReMiniMCRcFixXsec_GMCPhPtWt_tightzpt50_puWeightsummer16_muoneg_gjet_metfilter_unblind_mu_log_1pb.pdf}
\caption{${p_{T}}^{miss}$ distributions for electron (left) and muon (right) channels
comparing data and background.}
\label{fig:gjet_met_wide}
\end{figure}

\begin{figure}[htbp!]
\centering
\includegraphics[width=0.46\linewidth,page=18]{figures/ReMiniSummer16_DT_PhReMiniMCRcFixXsec_GMCPhPtWt_tightzpt50_puWeightsummer16_muoneg_gjet_metfilter_unblind_el_log_1pb.pdf}
\includegraphics[width=0.46\linewidth,page=18]{figures/ReMiniSummer16_DT_PhReMiniMCRcFixXsec_GMCPhPtWt_tightzpt50_puWeightsummer16_muoneg_gjet_metfilter_unblind_mu_log_1pb.pdf}
\caption{${p_{T}}^{miss}$ distributions for electron (left) and muon (right) channels
comparing data and background.}
\label{fig:gjet_met_narrow}
\end{figure}

\begin{figure}[htbp!]
\centering
\includegraphics[width=0.46\linewidth,page=22]{figures/ReMiniSummer16_DT_PhReMiniMCRcFixXsec_GMCPhPtWt_tightzpt50_puWeightsummer16_muoneg_gjet_metfilter_unblind_el_log_1pb.pdf}
\includegraphics[width=0.46\linewidth,page=22]{figures/ReMiniSummer16_DT_PhReMiniMCRcFixXsec_GMCPhPtWt_tightzpt50_puWeightsummer16_muoneg_gjet_metfilter_unblind_mu_log_1pb.pdf}
\caption{${p_{T}}^{miss}_\parallel$ distributions for electron (left) and muon (right)
channels comparing data and background.}
\label{fig:gjet_metpara}
\end{figure}

\begin{figure}[htbp!]
\centering
\includegraphics[width=0.46\linewidth,page=23]{figures/ReMiniSummer16_DT_PhReMiniMCRcFixXsec_GMCPhPtWt_tightzpt50_puWeightsummer16_muoneg_gjet_metfilter_unblind_el_log_1pb.pdf}
\includegraphics[width=0.46\linewidth,page=23]{figures/ReMiniSummer16_DT_PhReMiniMCRcFixXsec_GMCPhPtWt_tightzpt50_puWeightsummer16_muoneg_gjet_metfilter_unblind_mu_log_1pb.pdf}
\caption{${p_{T}}^{miss}_\perp$ distributions for electron (left) and muon (right)
channels comparing data and background.}
\label{fig:gjet_metperp}
\end{figure}

\clearpage
\section{Plots in the Signal Region}
Figure~\ref{fig:SR_gjet_rho} to \ref{fig:SR_gjet_metperp} shows the data vs background plots in the signal region, with all the backgrounds modeled in the method discussed above. 
\begin{figure}[htbp!]
\centering
\includegraphics[width=0.46\linewidth, page=1]{figures/ReMiniSummer16_DT_PhReMiniMCRcFixXsec_GMCPhPtWt_SRdPhiGT0p5_puWeightsummer16_muoneg_gjet_metfilter_unblind_el_log_1pb.pdf}
\includegraphics[width=0.46\linewidth, page=1]{figures/ReMiniSummer16_DT_PhReMiniMCRcFixXsec_GMCPhPtWt_SRdPhiGT0p5_puWeightsummer16_muoneg_gjet_metfilter_unblind_mu_log_1pb.pdf}
\caption{$\rho$ distributions for electron (left) and muon (right) channels
comparing data and background, in SR.}
\label{fig:SR_gjet_rho}
\end{figure}


\begin{figure}[htbp!]
\centering
\includegraphics[width=0.46\linewidth, page=2]{figures/ReMiniSummer16_DT_PhReMiniMCRcFixXsec_GMCPhPtWt_SRdPhiGT0p5_puWeightsummer16_muoneg_gjet_metfilter_unblind_el_log_1pb.pdf}
\includegraphics[width=0.46\linewidth, page=2]{figures/ReMiniSummer16_DT_PhReMiniMCRcFixXsec_GMCPhPtWt_SRdPhiGT0p5_puWeightsummer16_muoneg_gjet_metfilter_unblind_mu_log_1pb.pdf}
\caption{N reco vertices distributions for electron (left) and muon (right)
channels comparing data and background, in SR.}
\label{fig:SR_gjet_nvtx}
\end{figure}


\begin{figure}[htbp!]
\centering
\includegraphics[width=0.46\linewidth,page=3]{figures/ReMiniSummer16_DT_PhReMiniMCRcFixXsec_GMCPhPtWt_SRdPhiGT0p5_puWeightsummer16_muoneg_gjet_metfilter_unblind_el_log_1pb.pdf}
\includegraphics[width=0.46\linewidth,page=3]{figures/ReMiniSummer16_DT_PhReMiniMCRcFixXsec_GMCPhPtWt_SRdPhiGT0p5_puWeightsummer16_muoneg_gjet_metfilter_unblind_mu_log_1pb.pdf}
\caption{$m_T$ distributions for electron (left) and muon (right) channels
comparing data and background, 
wide mass window, in SR}
\label{fig:SR_gjet_mt_wide}
\end{figure}

\begin{figure}[htbp!]
\centering
\includegraphics[width=0.46\linewidth,page=5]{figures/ReMiniSummer16_DT_PhReMiniMCRcFixXsec_GMCPhPtWt_SRdPhiGT0p5_puWeightsummer16_muoneg_gjet_metfilter_unblind_el_log_1pb.pdf}
\includegraphics[width=0.46\linewidth,page=5]{figures/ReMiniSummer16_DT_PhReMiniMCRcFixXsec_GMCPhPtWt_SRdPhiGT0p5_puWeightsummer16_muoneg_gjet_metfilter_unblind_mu_log_1pb.pdf}
\caption{$m_T$ distributions for electron (left) and muon (right) channels
comparing data and background,
narrow mass window, in SR}
\label{fig:SR_gjet_mt_narrow}
\end{figure}

\begin{figure}[htbp!]
\centering
\includegraphics[width=0.46\linewidth,page=8]{figures/ReMiniSummer16_DT_PhReMiniMCRcFixXsec_GMCPhPtWt_SRdPhiGT0p5_puWeightsummer16_muoneg_gjet_metfilter_unblind_el_log_1pb.pdf}
\includegraphics[width=0.46\linewidth,page=8]{figures/ReMiniSummer16_DT_PhReMiniMCRcFixXsec_GMCPhPtWt_SRdPhiGT0p5_puWeightsummer16_muoneg_gjet_metfilter_unblind_mu_log_1pb.pdf}
\caption{Z mass distributions for electron (left) and muon (right) channels
comparing data and background, in SR.}
\label{fig:SR_gjet_mz}
\end{figure}

\begin{figure}[htbp!]
\centering
\includegraphics[width=0.46\linewidth,page=9]{figures/ReMiniSummer16_DT_PhReMiniMCRcFixXsec_GMCPhPtWt_SRdPhiGT0p5_puWeightsummer16_muoneg_gjet_metfilter_unblind_el_log_1pb.pdf}
\includegraphics[width=0.46\linewidth,page=9]{figures/ReMiniSummer16_DT_PhReMiniMCRcFixXsec_GMCPhPtWt_SRdPhiGT0p5_puWeightsummer16_muoneg_gjet_metfilter_unblind_mu_log_1pb.pdf}
\caption{Z $p_T$ distributions for electron (left) and muon (right) channels
comparing data and background,
wide binning, in SR.}
\label{fig:SR_gjet_zpt_wide}
\end{figure}

\begin{figure}[htbp!]
\centering
\includegraphics[width=0.46\linewidth,page=10]{figures/ReMiniSummer16_DT_PhReMiniMCRcFixXsec_GMCPhPtWt_SRdPhiGT0p5_puWeightsummer16_muoneg_gjet_metfilter_unblind_el_log_1pb.pdf}
\includegraphics[width=0.46\linewidth,page=10]{figures/ReMiniSummer16_DT_PhReMiniMCRcFixXsec_GMCPhPtWt_SRdPhiGT0p5_puWeightsummer16_muoneg_gjet_metfilter_unblind_mu_log_1pb.pdf}
\caption{Z $p_T$ distributions for electron (left) and muon (right) channels
comparing data and background, 
narrow binning, in SR.}
\label{fig:SR_gjet_zpt_narrow}
\end{figure}

\begin{figure}[htbp!]
\centering
\includegraphics[width=0.46\linewidth,page=16]{figures/ReMiniSummer16_DT_PhReMiniMCRcFixXsec_GMCPhPtWt_SRdPhiGT0p5_puWeightsummer16_muoneg_gjet_metfilter_unblind_el_log_1pb.pdf}
\includegraphics[width=0.46\linewidth,page=16]{figures/ReMiniSummer16_DT_PhReMiniMCRcFixXsec_GMCPhPtWt_SRdPhiGT0p5_puWeightsummer16_muoneg_gjet_metfilter_unblind_mu_log_1pb.pdf}
\caption{${p_{T}}^{miss}$ distributions for electron (left) and muon (right) channels
comparing data and background, in SR.}
\label{fig:SR_gjet_met_wide}
\end{figure}

\begin{figure}[htbp!]
\centering
\includegraphics[width=0.46\linewidth,page=18]{figures/ReMiniSummer16_DT_PhReMiniMCRcFixXsec_GMCPhPtWt_SRdPhiGT0p5_puWeightsummer16_muoneg_gjet_metfilter_unblind_el_log_1pb.pdf}
\includegraphics[width=0.46\linewidth,page=18]{figures/ReMiniSummer16_DT_PhReMiniMCRcFixXsec_GMCPhPtWt_SRdPhiGT0p5_puWeightsummer16_muoneg_gjet_metfilter_unblind_mu_log_1pb.pdf}
\caption{${p_{T}}^{miss}$ distributions for electron (left) and muon (right) channels
comparing data and background, in SR.}
\label{fig:SR_gjet_met_narrow}
\end{figure}

\begin{figure}[htbp!]
\centering
\includegraphics[width=0.46\linewidth,page=22]{figures/ReMiniSummer16_DT_PhReMiniMCRcFixXsec_GMCPhPtWt_SRdPhiGT0p5_puWeightsummer16_muoneg_gjet_metfilter_unblind_el_log_1pb.pdf}
\includegraphics[width=0.46\linewidth,page=22]{figures/ReMiniSummer16_DT_PhReMiniMCRcFixXsec_GMCPhPtWt_SRdPhiGT0p5_puWeightsummer16_muoneg_gjet_metfilter_unblind_mu_log_1pb.pdf}
\caption{${p_{T}}^{miss}_\parallel$ distributions for electron (left) and muon (right)
channels comparing data and background, in SR.}
\label{fig:SR_gjet_metpara}
\end{figure}


\begin{figure}[htbp!]
\centering
\includegraphics[width=0.46\linewidth,page=23]{figures/ReMiniSummer16_DT_PhReMiniMCRcFixXsec_GMCPhPtWt_SRdPhiGT0p5_puWeightsummer16_muoneg_gjet_metfilter_unblind_el_log_1pb.pdf}
\includegraphics[width=0.46\linewidth,page=23]{figures/ReMiniSummer16_DT_PhReMiniMCRcFixXsec_GMCPhPtWt_SRdPhiGT0p5_puWeightsummer16_muoneg_gjet_metfilter_unblind_mu_log_1pb.pdf}
\caption{${p_{T}}^{miss}_\perp$ distributions for electron (left) and muon (right)
channels comparing data and background, in SR.}
\label{fig:SR_gjet_metperp}
\end{figure}

