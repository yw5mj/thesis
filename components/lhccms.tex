
This analysis is based on the 2016 data collected in the CMS experiment. The Compact Muon Solenoid (CMS) experiment is a detector built on the Large Hadron Collider (LHC) located at CERN, Switzerland. This chapter gives an overview of LHC and CMS experiment.

\section{The Large Hadron Collider} 
The Large Hadron Collider (LHC) is the world most powerful particle accelerator and collider. It was built by the European Organization for Nuclear Research (CERN) between 1998 and 2008, in the tunnel of its predecessor, the Large Electron Positron Collider (LEP), with a circumfence of 27 km and as deep as 175 metres (574 ft) beneath the France–Switzerland border near Meyrin, Geneva. 

\vspace{0.3cm}
The LHC is designed mainly for proton-proton collisions, but can also be used to produce ion collisions. In this thesis only proton collisions are discussed. The first long proton run was performed in 2010, starting with the center of mass energy of 7 TeV, or 3.5 TeV per beam. After a short break at the end of 2011 the beam energy was increased from 3.5 TeV to 4 TeV (8 TeV center of mass energy). The period of these 7TeV/8TeV runs are often referred to as LHC Run I, and were ended early 2013 as the LHC came to the shut down period. The shut down period last for 2 years mainly for the LHC upgrading, which enabled collisions at 13 TeV. The Run II data taking started on 3 June 2015, with the center of mass energy of 13 TeV. 

\vspace{0.3cm}
The process of particle acceleration begins from a simple tank of hydrogen as the source of protons, which are progressively accelerated to higher energies in sequential machines ending at the LHC. A diagram of the CERN accelerator complex is shown in Figure~\ref{fig:lhc_lhc}. The hydrogen atoms are first accelerated by a linear accelerator, Linac2, after being stripped of their electrons. In Linac2, proton beam is accelerated by electric fields generated by alternatingly charged cylindrical conductors, and concentrated by magnetic fields generated by superconducting quadrupole magnets. The protons exit Linac2 with the energy of 50 MeV and enter the Proton Synchrotron Booster (PSB), where the particles are further accelerated to 1.4 GeV.  then the protons are transferred to the Proton Synchrotron (PS) with a circumfence of 628m where the energy of the protons is increased to 25 GeV. Following is the 7km long Super Proton Synchrotron (SPS) as the final step of the acceleration chain before the beams are injected into the LHC ring. There the protons are accelerated to an energy of 450 GeV.

\begin{figure}[htbp]
\begin{center}
\includegraphics[width=0.72\linewidth]{figures/lhc_lhc.pdf}
\caption{CERN accelerator complex including the four main experiments and the injection chain}
\label{fig:lhc_lhc}
\end{center}
\end{figure}

\vspace{0.3cm}
The beams are injected into the LHC rings for the final acceleration before collisions. There the protons reach the energy of 6.5 TeV after about 20 minutes' acceleration. Near 10,000 magnets are used on the LHC. Among them over 1000 superconducting dipole magnets are used to produce a magnetic field of 8.3T and bend the proton beams onto the circular trajectory. Besides, superconducting quadruple magnets are used to focues the beam, and the remaining sextupole and higher order magnets are used to correct the beam chromaticity.

\vspace{0.3cm}
Inside LHC particle collisions can happen at 4 interaction points of the tunnel, which correspond to 4 experiments: CMS, ALICE, ATLAS and LHCb. The ALICE experiment is designed to study the quark-gluon plasma using data collected during the heavy ion operation on the LHC. These measurements are designed to draw conclusions about the initial state of the universe. LHCb focuses on precisely measuring B-meson decays and CP-violating processes. CMS and ATLAS are the two general purpose experiments at the LHC build for studying a broad range of physics processes. These studies include precision measurements of Standard Model processes and parameters, thereby deepening our knowledge and understanding of the Standard Model. In addition, major fields of study are searches for the Higgs bosons and study of their properties and searches for physics beyond the Standard Model.

\vspace{0.3cm}
The number of collisions generated at the LHC is proportional to the cross section for proton-proton interactions and the integrated luminosity, and can be written as $N=\sigma \times L$. Here $L$ is the integrated luminosity and is defined as the integral of instantaneous luminosity (denoted by $\mathcal{L}$) over time, as shown in Equation~\ref{eqn:lhc_intelumi}.
\begin{equation}
L=\int \mathcal{L}dt
\label{eqn:lhc_intelumi}
\end{equation}
The maximal instantaneos luminosity of LHC in 2016 reached 15.30 Hz/nb, and the total integrated luminosity delievered by LHC in 2016 is 40.82 $fb^{-1}$, within which 37.76 $fb^{-1}$ is recorded by CMS and 35.9 $fb^{-1}$ is good for physics. Figure~\ref{fig:lhc_lumi2016} shows the development of instantaneos and integrated luminosity during year of 2016.

\begin{figure}[htbp]
\begin{center}
\includegraphics[width=0.49\linewidth, page=1]{figures/lhc_lumi2016.pdf}
\includegraphics[width=0.49\linewidth, page=2]{figures/lhc_lumi2016.pdf}
\caption{Daily peak instantaneos luminosity (left) and integrated luminosity (right) development of 2016 LHC proton-proton collisions.}
\label{fig:lhc_lumi2016}
\end{center}
\end{figure}

\section{The Compact Muon Solenoid (CMS) Experiment}
As one of the two general purpose detectors at the LHC, the Compact Muon Solenoid (CMS) detector\cite{lhc_cmsatcern}, sitting 100 meters underground at the LHC interaction point opposite the CERN site, in the French village of Cessy, measures 15 meters tall, 22 meters long. And the weight of 14,000 tons makes the CMS detector the heaviest detector in the world. The CMS detector is designed as a barrel like detector around the interaction point of the proton beams delivered by the LHC, consisting of multiple layers of subdetectors. Like a barrel, the detector can be seperated into two parts: the central part, ofter referred to as "barrel", and the two sections closing out the barrel on the two end of the detector named as "endcaps". One characteristic of the CMS detector is its strong axial magnetic field at the strength of 3.8 T, generated by a superconducting solenoid magnet surrounding the inner detector subsystems.

Figure~\ref{fig:lhc_cmsstructure} shows the structure and components of the CMS detector.
\begin{figure}[htbp]
\begin{center}
\includegraphics[width=0.7\linewidth, page=1]{figures/lhc_cmsstructure.pdf}
\includegraphics[width=0.7\linewidth, page=2]{figures/lhc_cmsstructure.pdf}
\caption{Illustration of CMS detector structure and subdetectors as components in a cut-out quadrant view(upper), and a cross sectional slice view as well as the trajectories of various particles from p-p collisions in the detector (lower).}
\label{fig:lhc_cmsstructure}
\end{center}
\end{figure}

\vspace{0.3cm}
Two sets of coordinate systems are widely used in CMS. One is an orthogonal right handed coordinate system, with the origin sitting at the nominal collision point inside the experiment. The x-axis points radially toward the center of the LHC ring, the y-axis points upward perpendicular to the LHC ring plane, and the z-axis correspondingly points counterclockwise along the beam pipe. Apart from the orthogonal coordinate system, a spherical coordinate system is used due to the cylindrically symmetric design of the detector. Different from the regular spherical coordinate with parameter $\phi$ denoting the azimuthal angle to the x axis in the x-y plane and parameter $\theta$ denoting the polar angle measured from the z axis, this coordinate in CMS keeps the $\phi$ parameter but use the pseudorapidity $\eta$ defined in Equationin~\ref{eqn:lhc_eta} in the place of $\theta$.
\begin{equation}
\eta = -ln[tan(\frac{\theta}{2})]
\label{eqn:lhc_eta}
\end{equation}

Based on this spherical coordinate, parameter $\Delta R$ is introduced as a measurement of the distance between two particles, and defined as $\Delta R=\sqrt{\Delta\eta^2 + \Delta\phi^2}$
.
\subsection{Inner Tracking System} 
The innter tracking system\cite{lhc_trackerdesign} is the innermost component of the CMS detector. The main functionality of the innter tracking system is to precisely measure the trajectory of charged particles such as charged leptons and hadrons. It consists of an inner pixel detector and a strip tracker. Figure~\ref{fig:lhc_trackerbarrel} shows the structure of the tracking system
\begin{figure}[htbp]
\begin{center}
\includegraphics[width=0.7\linewidth, page=1]{figures/lhc_trackerbarrel.pdf}
\includegraphics[width=0.7\linewidth, page=2]{figures/lhc_trackerbarrel.pdf}
\caption{Structure of CMS tracking system in the plane parallel(upper) and perpendicular(lower) to the z axis.}
\label{fig:lhc_trackerbarrel}
\end{center}
\end{figure}

\subsubsection{inner pixel detector}
The inner pixel detector, consisting of 3 cylindrical layers with minimal radius of 4 cm and disks on each end as endcaps, is the closest part to the interaction point of the experiment, which makes it vital in reconstructing the tracks of very short-lived particles. But being close to the interactions also means an enormous particle flux. Therefore the pixel detector's design was driven by the goal of getting the best track position resolution possible while also being vey radiation torlerant. The inner pixel detector contains 65 million silicon pixel sensors each with a dimension of $100\mu m \times 150\mu m$. These pixel sensors were built using high dose n-implants on a high resistance n-substrate, which ensured high signal collection efficiency with only moderate bias voltages even after high doses of radiation.
\subsubsection{strip tracker}
After passing through the pixel detector, particles reach the outer silicon strip tracker at radius of 130 cm. The strip detector consists of the Tracker Inner Barrel (TIB), Tracker Inner Disks (TID), Tracker Outer Barrel (TOB), and the Tracker End Caps (TEC). TIB contains 4 layers of silicon sensors, among which the inner 2 layers are built with double sided sensors while the other 2 layers are single sided. The TID, placed on each end of TIB as the endcaps, is composed of 2 sets of disks of sensors, with 3 disks in each set. Surrounding the inner tracker is the TOB, consisting of 6 layers. The inner 2 layers carry double sided sensors. Finally the endcaps(TEC) close off the tracking system, with 9 disks of sensors on each side. 

\vspace{0.3cm}
The CMS tracking system covers the detector region up to $|\eta| = 2.5$ and has a resolution of up to 10 $\mu m$ in the x-y direction and 20 $\mu m$ in the z direction. 
\subsection{Electromagnetic Calorimeter} 
The Electromagnetic Calorimeter (ECAL)\cite{lhc_ecaldesign} is designed mainly for the measurement of the energy of photons and electrons. It sits between the tracker and the Hadron Calorimeter, covering $\eta$ range from -3.0 to 3.0. ECAL is composed of the ECAL Barrel (EB), ECAL Endcap (EE) and ECAL Preshower (ES). Figure~\ref{fig:lhc_ecal} shows the structure of ECAL.
\begin{figure}[htbp]
\begin{center}
\includegraphics[width=0.7\linewidth]{figures/lhc_ecal.pdf}
\caption{Structure of CMS tracking system in the y-z plane.}
\label{fig:lhc_ecal}
\end{center}
\end{figure}

\subsubsection{ECAL Barrel and Endcap}
ECAL Barrel and Endcap contains near 80,000 lead tungstate (PbWO4) crystals. These crystals are primarily made of metal with high density, good radiation torlerance and short radiation length. This material can precisely produce scintillation light responding to fast and small photon showers, and is relatively compact due to its high density. However, the drawback the crystal is that its yiled of light strongly depens on temperature. The nominal operating temperature of the ECAL is maintained at $18^{\circ}C$, with variation within $0.1^{\circ}C$, for desired energy measurement resolution. 

\vspace{0.3cm}
The EB consists of over 60,000 crystals and covers $|\eta|$ up to 1.479. Avalanche photodiodes (APD) are mounted on the rear face of these crystals to collect the scintillation light and amplify the signal. 

\vspace{0.3cm}
Each of the two endcaps contians over 7,000 crystals. The crystals are grouped in $5\times 5$ arrays and are referred to as "super crystals". Vacuum phototriodes (VPT) are glued to these crystals for signal collection and amplification, similar to the APDs to the EB. The EE covers $|\eta|$ range between 1.479 and 3.0. 

\subsubsection{ECAL Preshower}

\subsection{Hadronic Calorimeter} 
\subsection{Muon Chambers} 
\subsection{Data Acquisition and Trigger System} 
