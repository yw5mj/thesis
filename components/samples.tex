\section{Data and Triggers}
\subsection{Data}
This analysis uses a data sample recorded by the CMS detector during 2016 with an integrated luminosity of 35.9 fb$^{-1}$. The Primary Datasets (PD) used are listed in Table~\ref{tab:samples_data}. 
\begin{table}[htbp]
  \begin{center}
    \caption{Datasets used in the analysis. \label{tab:samples_data}}
    \begin{tabular}{c}
      \hline\hline
      Dataset   \\
      \hline
      \texttt{/PD/Run2016B-03Feb2017\_ver1-v1} \\ 
      \texttt{/PD/Run2016B-03Feb2017\_ver2-v2}  \\
      \texttt{/PD/Run2016C-03Feb2017-v1}  \\
      \texttt{/PD/Run2016D-03Feb2017-v1}  \\
      \texttt{/PD/Run2016E-03Feb2017-v1}  \\
      \texttt{/PD/Run2016F-03Feb2017-v1}  \\
      \texttt{/PD/Run2016G-03Feb2017v1}  \\
      \texttt{/PD/Run2016H-03Feb2017\_ver2-v1}  \\
      \texttt{/PD/Run2016H-03Feb2017\_ver3-v1}  \\
      \hline
      \textbf{Primary datasets (PD)}: \\
      \texttt{SingleElectron, SingleMuon, SinglePhoton, and MuEG}\\ \hline
      \hline\hline
    \end{tabular}
  \end{center}
\end{table}

\subsection{High Level Triggers}
In this analysis the single lepton (electron or muon) $High\,Level$ triggers without pre-selections on isolation are required for the signal data selection. The single lepton triggers are preferred over the di-lepton triggers, in the consideration of signal protection: the two leptons can be very adjacent to each other and .single photon $High\,Level$ triggers for the $photon+jets$ event selection used in the data-driven Drell-Yan ($Z+jets$) background modeling. The detailed HLT paths are listed in Table~\ref{tab:triggerPaths}. 

\begin{table}[htbp!]
   \caption{Trigger paths used in 2016 collision data.
   \label{tab:triggerPaths}}
%\scriptsize
   \centering
   \begin{tabular}{ c }
\hline\hline
Muon  \\ \texttt{HLT\_Mu50\_v* OR HLT\_TkMu50\_v*}  \\ \hline
Electron \\ \texttt{HLT\_Ele115\_CaloIdVT\_GsfTrkIdT\_v*} \\ \hline
Photon \\ \texttt{HLT\_PhotonPT\_R9Id90\_HE10\_IsoM\_v*},  for \texttt{PT} = 22,30,36,50,75,90,120,165 GeV \\  
\hline
\hline
    \end{tabular}
\end{table}

\section{Monte Carlo Simulation Samples}
The Monte Carlo (MC) simulation samples are used in this analysis, both to help modeling the SM background accurately to confront the data selected from the experiment, and to help predicting the likelihood of the existence of the Bulk Graviton signal. Therefore, simulation samples for both SM background and signal are required. The Parton Distribution Functions (PDF), which discribe the fraction of momentum a parton carries within a proton, are modeled using the NNPDF3.0~\cite{sample_nnpdf30} parametrization, based on the recommendation of the PDF4LHC team~\cite{sample_pdf4lhc}. The simulation process can be seen in 3 stages: the simulation for the core physics process including the subsequent decays, handled by matrix element generators; the simulation for the hadron showering and hadronization processes, handled by parton shower generators, which is Pythia8.212~\cite{sample_pythia} for this analysis; and the simulation for the experimental observation from the detector, based on GEANT4~\cite{sample_geant4}. The MC samples used in this analysis are in the format of standard CMS Mini-AOD format~\cite{sample_miniaod}, in the campaign of RunIISummer16.

\subsection{Standard Model Samples}
The SM background in this analysis is mostly dominated by the $Z+jets$ process. The other main source of the background besides $Z+jets$ are the nonresonant production of $\ell$ and ${p_{T}}^{miss}$ final states, including $t\bar{t}$ and $WW$ processes, and the resonant background from SM production of diboson events, composed mainly of $ZZ$ and $WZ$ processes. 
\subsection{Signal Samples}




